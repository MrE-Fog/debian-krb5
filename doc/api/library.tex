\documentclass[twoside]{article}
\usepackage{fixunder,functions,fancyheadings}
\usepackage{krb5idx}
%\usepackage{hyperref}

%\hypersetup{letterpaper,
%    bookmarks=true,
%pdfpagemode=UseOutlines,
%}

%
%\setlength{\oddsidemargin}{1in}
%\setlength{\evensidemargin}{1.00in}
%\setlength{\textwidth}{6.5in}
\setlength{\oddsidemargin}{0in}
\setlength{\evensidemargin}{1.50in}
\setlength{\textwidth}{5.25in}
\setlength{\marginparsep}{0.0in}
\setlength{\marginparwidth}{1.95 in}
\setlength{\topmargin}{-.5in}
\setlength{\textheight}{9in}
\setlength{\parskip}{.1in}
\setlength{\parindent}{2em}
\setlength{\footrulewidth}{0.4pt}
\setlength{\plainfootrulewidth}{0.4pt}
\setlength{\plainheadrulewidth}{0.4pt}
\makeindex
\newif\ifdraft
\draftfalse
%
% Far, far too inconvenient... it's still very draft-like anyway....
%   [tytso:19900921.0018EDT]
%
%\typein{Draft flag? (type \noexpand\draftfalse<CR> if not draft...)}
\ifdraft
\pagestyle{fancyplain}
\addtolength{\headwidth}{\marginparsep}
\addtolength{\headwidth}{\marginparwidth}
\makeatletter
\renewcommand{\sectionmark}[1]{\markboth {\uppercase{\ifnum \c@secnumdepth >\z@
    \thesection\hskip 1em\relax \fi #1}}{}}%
\renewcommand{\subsectionmark}[1]{\markright {\ifnum \c@secnumdepth >\@ne
          \thesubsection\hskip 1em\relax \fi #1}}
\makeatother
\lhead[\thepage]{\fancyplain{}{\sl\rightmark}}
\rhead[\fancyplain{}{\sl\rightmark}]{\thepage}
\lfoot[]{{\bf DRAFT---DO NOT REDISTRIBUTE}}
\rfoot[{\bf DRAFT---DO NOT REDISTRIBUTE}]{}
\cfoot{\thepage}
\else\pagestyle{headings}\fi

\def\internalfunc{NOTE: This is an internal function, which is not
necessarily intended for use by application programs.  Its interface may
change at any time.\par}

%nlg- time to make this a real document

\title{\Huge Kerberos V5 application programming library}
\date{\ifdraft \\ {\Large DRAFT---}\fi\today}
\author{MIT Information Systems}

\begin{document}
\maketitle
\tableofcontents

%\thispagestyle{empty}
%\begin{center}
%{\Huge Kerberos V5 application programming library}
%\ifdraft \\ {\Large DRAFT---\today}\fi
%\end{center}

\section{Introduction}
	This document describes the routines that make up the Kerberos
V5 application programming interface.  It is geared towards
programmers who already have a basic familiarity with Kerberos and are
in the process of including Kerberos authentication as part of 
applications being developed.

	The function descriptions included are up to date, even if the
description of the functions may be hard to understand for the novice
Kerberos programmer.

\subsection{Acknowledgments}


The Kerberos model is based in part on Needham and Schroeder's trusted
third-party authentication protocol and on modifications suggested by
Denning and Sacco.  The original design and implementation of Kerberos
Versions 1 through 4 was the work of Steve Miller of Digital Equipment
Corporation and Clifford Neuman (now at the Information Sciences
Institute of the University of Southern California), along with Jerome
Saltzer, Technical Director of Project Athena, and Jeffrey Schiller,
MIT Campus Network Manager.  Many other members of Project Athena have
also contributed to the work on Kerberos.  Version 4 is publicly
available, and has seen wide use across the Internet.

Version 5 (described in this document) has evolved from Version 4 based
on new requirements and desires for features not available in Version 4.

%nlg- a bunch more probably needs to be added here to credit all
%those that have contributed to V5 -nlg

\subsection{Kerberos Basics}

Kerberos performs authentication as a trusted third-party
authentication service by using conventional (shared secret
key\footnote{ {\em Secret} and {\em private} are often used
interchangeably in the literature.  In our usage, it takes two (or
more) to share a secret, thus a shared DES key is a {\em secret} key.
Something is only private when no one but its owner knows it.  Thus,
in public key cryptosystems, one has a public and a {\em private} key.
}) cryptography.  Kerberos provides a means of verifying the
identities of principals, without relying on authentication by the
host operating system, without basing trust on host addresses, without
requiring physical security of all the hosts on the network, and under
the assumption that packets traveling along the network can be read,
modified, and inserted at will.

When integrating Kerberos into an application it is important to
review how and when Kerberos functions are used to ensure that the
application's design does not compromise the authentication.  For
instance, an application which uses Kerberos' functions only upon the
{\em initiation} of a stream-based network connection, and assumes the
absence of any active attackers who might be able to ``hijack'' the
stream connection.

%{\Huge nlg- It would be nice to include more examples here of common
%mistakes one can make in designing kerberized systems -nlg}

The Kerberos protocol code libraries, whose API is described in this
document, can be used to provide encryption to any application.  In
order to add authentication to its transactions, a typical network
application adds one or two calls to the Kerberos library, which
results in the transmission of the necessary messages to achieve
authentication.

The two methods for obtaining credentials, the initial ticket exchange
and the ticket granting ticket exchange, use slightly different
protocols and require different API routines.  The basic difference an
API programmer will see is that the initial request does not require a
ticket granting ticket (TGT) but does require the client's secret key
because the reply is sent back encrypted in the client's secret key.
Usually this request is for a TGT and TGT based exchanges are used
from then on.  In a TGT exchange the TGT is sent as part of the
request for tickets and the reply is encrypted in the session key from
the TGT.  For example, once a user's password is used to obtain a TGT,
it is not required for subsequent TGT exchanges.

The reply consists of a ticket and a session key, encrypted either in
the user's secret key (i.e., password), or the TGT session key.  The
combination of a ticket and a session key is known as a set of {\em
credentials}.\footnote{In Kerberos V4, the ``ticket file'' was a bit of
a misnomer, since it contained both tickets and their associated session
keys.  In Kerberos V5, the ``ticket file'' has been renamed to be the
{\em credentials cache}.} An application client can use these
credentials to authenticate to the application server by sending the
ticket and an {\em authenticator} to the server.  The authenticator is
encrypted in the session key of the ticket, and contains the name of the
client, the name of the server, the time the authenticator was created.

In order to verify the authentication, the application server decrypts
the ticket using its service key, which is only known by the application
server and the Kerberos server.  Inside the ticket, the Kerberos server
had placed the name of the client, the name of the server, a DES key
associated with this ticket, and some additional information.  The
application server then uses the ticket session key to decrypt the
authenticator, and verifies that the information in the authenticator
matches the information in the ticket, and that the timestamp in the
authenticator is recent (to prevent reply attacks).  Since the session
key was generated randomly by the Kerberos server, and delivered only
encrypted in the service key, and in a key known only by the user, the
application server can be confident that user is really who he or she
claims to be, by virtue of the fact that the user was able to encrypt
the authenticator in the correct key.

To provide detection of both replay
attacks and message stream modification attacks, the integrity of all
the messages exchanged between principals can also be 
guaranteed\footnote{Using
\funcname{krb5_mk_safe} and \funcname{krb5_rd_safe} to create and
verify KRB5_SAFE messages} by generating and transmitting a
collision-proof checksum \footnote{aka cryptographic checksum,
elsewhere this is called a hash or digest function} of the client's
message, keyed with the session key.  Privacy and integrity of the
messages exchanged between principals can be secured\footnote{Using
\funcname{krb5_mk_priv} and \funcname{krb5_rd_priv} to create and
verify KRB5_PRIV messages} by encrypting the data to be passed using
the session key.

\subsubsection{The purpose of Realms}

The Kerberos protocol is designed to operate across organizational
boundaries.   Each organization wishing to run a Kerberos
server establishes its own {\em realm}.  The name of the realm in which a
client is registered is part of the client's name, and can be used by the
end-service to decide whether to honor a request.

By establishing {\em inter-realm} keys, the administrators of two
realms can allow a client authenticated in the local realm to use its
credentials remotely.  The exchange of inter-realm keys (a separate
key may be used for each direction) registers the ticket-granting
service of each realm as a principal in the other realm.  A client is
then able to obtain a ticket-granting ticket for the remote realm's
ticket-granting service from its local realm.  When that
ticket-granting ticket is used, the remote ticket-granting service
uses the inter-realm key (which usually differs from its own normal
TGS key) to decrypt the ticket-granting ticket, and is thus certain
that it was issued by the client's own TGS. Tickets issued by the
remote ticket-granting service will indicate to the end-service that
the client was authenticated from another realm.   


This method can be repeated to authenticate throughout an organization
across multiple realms.  To build a valid authentication
path\footnote{An {\em authentication path} is the sequence of
intermediate realms that are transited in communicating from one realm
to another.} to a distant realm, the local realm must share an
inter-realm key with an intermediate realm which
communicates\footnote{A realm is said to {\em communicate} with
another realm if the two realms share an inter-realm key} with either
the distant remote realm or yet another intermediate realm.

Realms are typically organized hierarchically.  Each realm shares a
key with its parent and a different key with each child.  If an
inter-realm key is not directly shared by two realms, the hierarchical
organization allows an authentication path to be easily constructed.
If a hierarchical organization is not used, it may be necessary to
consult some database in order to construct an authentication path
between realms.

Although realms are typically hierarchical, intermediate realms may be
bypassed to achieve cross-realm authentication through alternate
authentication paths\footnote{These might be established to make communication
between two realms more efficient}.  It is important for the
end-service to know which realms were transited when deciding how much
faith to place in the authentication process.  To facilitate this
decision, a field in each ticket contains the names of the realms that
were involved in authenticating the client.

\subsubsection{Fundamental assumptions about the environment}

Kerberos has certain limitations that should be kept in mind when
designing security measures:

\begin{itemize}
\item
Kerberos does not address ``Denial of service'' attacks.  There are
places in these protocols where an intruder can prevent an application
from participating in the proper authentication steps.  Detection and
solution of such attacks (some of which can appear to be not-uncommon
``normal'' failure modes for the system) is usually best left to
the human administrators and users.

\item
Principals must keep their secret keys secret.  If an intruder somehow
steals a principal's key, it will be able to masquerade as that
principal or impersonate any server to the legitimate principal.

\item
``Password guessing'' attacks are not solved by Kerberos.  If a user
chooses a poor password, it is possible for an attacker to
successfully mount an offline dictionary attack by repeatedly
attempting to decrypt, with successive entries from a dictionary,
messages obtained which are encrypted under a key derived from the
user's password.

\end{itemize}

\subsection{Glossary of terms}

Below is a list of terms used throughout this document.

\begin{description}
\item [Authentication] 
Verifying the claimed identity of a principal.

\item [Authentication header]
A record containing a Ticket and an Authenticator to be presented to a
server as part of the authentication process.

\item [Authentication path]
A sequence of intermediate realms transited in the authentication
process when communicating from one realm to another.

\item [Authenticator]
A record containing information that can be shown to
have been recently generated using the session key known only by the 
client and server.

\item [Authorization]
The process of determining whether a client may use a
service,  which objects the client is allowed to access, and the 
type of access allowed for each.

\item [Ciphertext]
The output of an encryption function.  Encryption transforms plaintext
into ciphertext.

\item [Client]
A process that makes use of a network service on behalf of a
user.  Note that in some cases a {\em Server} may itself be a client of
some other server (e.g. a print server may be a client of a file server).

\item [Credentials]
A ticket plus the secret session key necessary to
successfully use that ticket in an authentication exchange.

\item [KDC]
Key Distribution Center, a network service that supplies
tickets and temporary session keys; or an
instance of that service or the host on which it runs.
The KDC services both initial ticket and ticket-granting ticket
requests.
The initial ticket portion is sometimes referred to as the
Authentication Server (or service).
The ticket-granting ticket portion is sometimes referred to as the
ticket-granting server (or service).

\item [Kerberos]
Aside from the 3-headed dog guarding Hades, the name given
to Project Athena's authentication service, the protocol used by that
service, or the code used to implement the authentication service.

\item [Plaintext]
The input to an encryption function or the output of a decryption
function.  Decryption transforms ciphertext into plaintext.

\item [Principal]
A uniquely named client or server instance that participates in
a network communication.

\item [Principal identifier]
The name used to uniquely identify each different
principal.

\item [Seal]
To encipher a record containing several fields in such a way
that the fields cannot be individually replaced without either
knowledge of the encryption key or leaving evidence of tampering.

\item [Secret key]
An encryption key shared by a principal and the KDC,
distributed outside the bounds of the system, with a long lifetime.
In the case of a human user's principal, the secret key is derived from a
password.

\item [Server]
A particular Principal which provides a resource to network clients.

\item [Service]
A resource provided to network clients; often provided by more than one
server (for example, remote file service).

\item [Session key]
A temporary encryption key used between two principals,
with a lifetime limited to the duration of a single login
{\em session}.

\item [Sub-session key] 
A temporary encryption key used between two
principals, selected and exchanged by the principals using the session
key, and with a lifetime limited to the duration of a single
association.

\item [Ticket]
A record that helps a client authenticate itself to a server; it contains
the client's identity, a session key, a timestamp, and other
information, all sealed using the server's secret key.  It only serves to
authenticate a client when presented along with a fresh Authenticator.
\end{description}


\section{Useful KDC parameters to know about}
The following is a list of options which can be passed to the Kerberos
server (also known as the Key Distribution Center or KDC).  These
options affect what sort of tickets the KDC will return to the
application program.  The KDC options can be passed to
\funcname{krb5_get_in_tkt}, \funcname{krb5_get_in_tkt_with_password},
\funcname{krb5_get_in_tkt_with_skey}, and \funcname{krb5_send_tgs}. 


\begin{center}
\begin{tabular}{llc}
\multicolumn{1}{c}{Symbol}&\multicolumn{1}{c}{RFC}& Valid for \\
&\multicolumn{1}{c}{section}&get_in_tkt? \\ \hline
KDC_OPT_FORWARDABLE	& 2.6	& yes		\\
KDC_OPT_FORWARDED	& 2.6	&		\\
KDC_OPT_PROXIABLE	& 2.5	& yes		\\
KDC_OPT_PROXY		& 2.5	&		\\
KDC_OPT_ALLOW_POSTDATE	& 2.4	& yes		\\
KDC_OPT_POSTDATED	& 2.4	& yes		\\
KDC_OPT_RENEWABLE	& 2.3	& yes		\\
KDC_OPT_RENEWABLE_OK	& 2.7	& yes		\\
KDC_OPT_ENC_TKT_IN_SKEY	& 2.7	&		\\
KDC_OPT_RENEW		& 2.3	&		\\
KDC_OPT_VALIDATE	& 2.2	&		\\
\end{tabular}
\end{center}
\label{KDCOptions}

The following is a list of preauthentication methods which are supported
by Kerberos.  Most preauthentication methods are used by
krb5_get_in_tkt(), krb5_get_in_tkt_with_password(), and
krb5_get_in_tkt_with_skey(); at some sites, the Kerberos server can be
configured so that during the initial ticket transation, it will only
return encrypted tickets after the user has proven his or her identity
using a supported preauthentication mechanism.  This is done to make
certain password guessing attacks more difficult to carry out.



\begin{center}
\begin{tabular}{lcc}
\multicolumn{1}{c}{Symbol}&In & Valid for \\
&RFC?&get_in_tkt? \\ \hline
KRB5_PADATA_NONE		& yes	& yes	\\
KRB5_PADATA_AP_REQ		& yes	&	\\
KRB5_PADATA_TGS_REQ		& yes	&	\\
KRB5_PADATA_PW_SALT		& yes	&	\\
KRB5_PADATA_ENC_TIMESTAMP	& yes	& yes	\\
KRB5_PADATA_ENC_SECURID		&	& yes	\\
\end{tabular}
\end{center}
\label{padata-types}

KRB5_PADATA_TGS_REQ is rarely used by a programmer; it is used to pass
the ticket granting ticket to the Ticket Granting Service (TGS) during a
TGS transaction (as opposed to an initial ticket transaction).

KRB5_PW_SALT is not really a preauthentication method at all.  It is
passed back from the Kerberos server to application program, and it
contains a hint to the proper password salting algorithm which should be
used during the initial ticket exchange.

%The encription type can also be specified in
%\funcname{krb5_get_in_tkt}, however normally only one keytype is used
%in any one database.
%
%\begin{center}
%\begin{tabular}{llc}
%\multicolumn{1}{c}{Symbol}&\multicolumn{1}{c}{RFC}& Supported? \\
%& \multicolumn{1}{c}{section} &  \\ \hline
%ETYPE_NULL		& 6.3.1	& 	\\
%ETYPE_DES_CBC_CRC	& 6.3.2	& yes	\\
%ETYPE_DES_CBC_MD4	& 6.3.3	&	\\
%ETYPE_DES_CBC_MD5	& 6.3.4	&	\\
%ETYPE_RAW_DES_CBC	&	& yes	\\
%\end{tabular}
%\end{center}
%\label{etypes}




\section{Error tables}
\subsection{error_table krb5}

% $Source$
% $Author$

The Kerberos v5 library error code table follows.
Protocol error codes are ERROR_TABLE_BASE_krb5 + the protocol error
code number.  Other error codes start at ERROR_TABLE_BASE_krb5 + 128.

\begin{small}
\begin{tabular}{ll}
{\sc krb5kdc_err_none }&	 No error \\
{\sc krb5kdc_err_name_exp }& Client's entry in database has expired \\
{\sc krb5kdc_err_service_exp }& Server's entry in database has expired \\
{\sc krb5kdc_err_bad_pvno }& Requested protocol version not supported \\	
{\sc krb5kdc_err_c_old_mast_kvno }& \parbox[t]{2in}{Client's key is encrypted in an old master key} \\
{\sc krb5kdc_err_s_old_mast_kvno }& \parbox[t]{2in}{Server's key is encrypted in an old master key} \\
{\sc krb5kdc_err_c_principal_unknown }&  Client not found in Kerberos database \\
{\sc krb5kdc_err_s_principal_unknown }&  Server not found in Kerberos database \\
{\sc krb5kdc_err_principal_not_unique }&\parbox[t]{2in}{\raggedright{Principal has multiple entries in Kerberos database}} \\
{\sc krb5kdc_err_null_key }& Client or server has a null key \\
{\sc krb5kdc_err_cannot_postdate }& Ticket is ineligible for postdating \\
{\sc krb5kdc_err_never_valid }& \parbox[t]{2in}{Requested effective lifetime is negative or too short} \\
{\sc krb5kdc_err_policy }&	 KDC policy rejects request \\
{\sc krb5kdc_err_badoption }& KDC can't fulfill requested option \\
{\sc krb5kdc_err_etype_nosupp }& KDC has no support for encryption type \\
{\sc krb5kdc_err_sumtype_nosupp }& KDC has no support for checksum type \\
{\sc krb5kdc_err_padata_type_nosupp }&  KDC has no support for padata type \\
{\sc krb5kdc_err_trtype_nosupp }& KDC has no support for transited type \\
{\sc krb5kdc_err_client_revoked }& Clients credentials have been revoked \\
{\sc krb5kdc_err_service_revoked }& Credentials for server have been revoked \\
{\sc krb5kdc_err_tgt_revoked }& TGT has been revoked \\
{\sc krb5kdc_err_client_notyet }& Client not yet valid - try again later \\
{\sc krb5kdc_err_service_notyet }& Server not yet valid - try again later \\
{\sc krb5kdc_err_key_exp }&  	 Password has expired \\
{\sc krb5kdc_preauth_failed }&  	 Preauthentication failed \\
{\sc krb5kdc_err_preauth_require }&	Additional pre-authentication required \\
{\sc krb5kdc_err_server_nomatch }&	Requested server and ticket don't match \\
\multicolumn{2}{c}{error codes 27-30 are currently placeholders}\\

\end{tabular}

\begin{tabular}{ll}
{\sc krb5krb_ap_err_bad_integrity }&  Decrypt integrity check failed \\
{\sc krb5krb_ap_err_tkt_expired }& Ticket expired \\
{\sc krb5krb_ap_err_tkt_nyv }& Ticket not yet valid \\
{\sc krb5krb_ap_err_repeat }& Request is a replay \\
{\sc krb5krb_ap_err_not_us }& The ticket isn't for us \\
{\sc krb5krb_ap_err_badmatch }& Ticket/authenticator don't match \\
{\sc krb5krb_ap_err_skew }& Clock skew too great \\
{\sc krb5krb_ap_err_badaddr }& Incorrect net address \\
{\sc krb5krb_ap_err_badversion }& Protocol version mismatch \\
{\sc krb5krb_ap_err_msg_type }& Invalid message type \\
{\sc krb5krb_ap_err_modified }& Message stream modified \\
{\sc krb5krb_ap_err_badorder }& Message out of order \\
{\sc krb5placehold_43 }&	 KRB5 error code 43 \\
{\sc krb5krb_ap_err_badkeyver }& Key version is not available \\
{\sc krb5krb_ap_err_nokey }& Service key not available \\
{\sc krb5krb_ap_err_mut_fail }& Mutual authentication failed \\
{\sc krb5krb_ap_err_baddirection }& Incorrect message direction \\
{\sc krb5krb_ap_err_method }& Alternative authentication method required \\
{\sc krb5krb_ap_err_badseq }& Incorrect sequence number in message \\
{\sc krb5krb_ap_err_inapp_cksum }& Inappropriate type of checksum in message \\ 
\multicolumn{2}{c}{error codes 51-59 are currently placeholders} \\

{\sc krb5krb_err_generic }& Generic error (see e-text) \\
{\sc krb5krb_err_field_toolong }& Field is too long for this implementation \\
\multicolumn{2}{c}{error codes 62-127 are currently placeholders} \\
\end{tabular}

\begin{tabular}{ll}
{\sc krb5_libos_badlockflag }& Invalid flag for file lock mode \\
{\sc krb5_libos_cantreadpwd }& Cannot read password \\
{\sc krb5_libos_badpwdmatch }& Password mismatch \\
{\sc krb5_libos_pwdintr }&	 Password read interrupted \\
{\sc krb5_parse_illchar }&	 Illegal character in component name \\
{\sc krb5_parse_malformed }& Malformed representation of principal \\
{\sc krb5_config_cantopen }& Can't open/find configuration file \\
{\sc krb5_config_badformat }& Improper format of configuration file \\
{\sc krb5_config_notenufspace }& Insufficient space to return complete information \\
{\sc krb5_badmsgtype }&	 Invalid message type specified for encoding \\
{\sc krb5_cc_badname }&	 Credential cache name malformed \\
{\sc krb5_cc_unknown_type }& Unknown credential cache type  \\
{\sc krb5_cc_notfound }&	 Matching credential not found \\
{\sc krb5_cc_end }&		 End of credential cache reached \\
{\sc krb5_no_tkt_supplied }& Request did not supply a ticket \\
{\sc krb5krb_ap_wrong_princ }&	 Wrong principal in request \\
{\sc krb5krb_ap_err_tkt_invalid }& Ticket has invalid flag set \\
{\sc krb5_princ_nomatch }&	 Requested principal and ticket don't match \\
{\sc krb5_kdcrep_modified }& KDC reply did not match expectations \\
{\sc krb5_kdcrep_skew }&	Clock skew too great in KDC reply \\
{\sc krb5_in_tkt_realm_mismatch }&\parbox[t]{2.5 in}{Client/server realm
mismatch in initial ticket requst}\\

{\sc krb5_prog_etype_nosupp }& Program lacks support for encryption type \\
{\sc krb5_prog_keytype_nosupp }& Program lacks support for key type \\
{\sc krb5_wrong_etype }&	 Requested encryption type not used in message \\
{\sc krb5_prog_sumtype_nosupp }& Program lacks support for checksum type \\
{\sc krb5_realm_unknown }&	 Cannot find KDC for requested realm \\
{\sc krb5_service_unknown }&	Kerberos service unknown \\
{\sc krb5_kdc_unreach }&	 Cannot contact any KDC for requested realm \\
{\sc krb5_no_localname }&	 No local name found for principal name \\

%\multicolumn{1}{c}{some of these should be combined/supplanted by system codes} \\
\end{tabular}

\begin{tabular}{ll}
{\sc krb5_rc_type_exists }&	 Replay cache type is already registered \\
{\sc krb5_rc_malloc }&	 No more memory to allocate (in replay cache code) \\
{\sc krb5_rc_type_notfound }& Replay cache type is unknown \\
{\sc krb5_rc_unknown }&	 Generic unknown RC error \\
{\sc krb5_rc_replay }&	 Message is a replay \\
{\sc krb5_rc_io }&		 Replay I/O operation failed XXX \\
{\sc krb5_rc_noio }&	 \parbox[t]{3in}{Replay cache type does not support non-volatile storage} \\
{\sc krb5_rc_parse }& Replay cache name parse/format error \\
{\sc krb5_rc_io_eof }&	 End-of-file on replay cache I/O \\
{\sc krb5_rc_io_malloc }& \parbox[t]{3in}{No more memory to allocate (in replay cache I/O code)}\\
{\sc krb5_rc_io_perm }&	 Permission denied in replay cache code \\
{\sc krb5_rc_io_io }&	 I/O error in replay cache i/o code \\
{\sc krb5_rc_io_unknown }&	 Generic unknown RC/IO error \\
{\sc krb5_rc_io_space }& Insufficient system space to store replay information \\
{\sc krb5_trans_cantopen }&	 Can't open/find realm translation file \\
{\sc krb5_trans_badformat }& Improper format of realm translation file \\
{\sc krb5_lname_cantopen }&	 Can't open/find lname translation database \\
{\sc krb5_lname_notrans }&	 No translation available for requested principal \\
{\sc krb5_lname_badformat }& Improper format of translation database entry \\
{\sc krb5_crypto_internal }& Cryptosystem internal error \\
{\sc krb5_kt_badname }&	 Key table name malformed \\
{\sc krb5_kt_unknown_type }& Unknown Key table type  \\
{\sc krb5_kt_notfound }&	 Key table entry not found \\
{\sc krb5_kt_end }&		 End of key table reached \\
{\sc krb5_kt_nowrite }&	 Cannot write to specified key table \\
{\sc krb5_kt_ioerr }&	 Error writing to key table \\
{\sc krb5_no_tkt_in_rlm }&	 Cannot find ticket for requested realm \\
{\sc krb5des_bad_keypar }&	 DES key has bad parity \\
{\sc krb5des_weak_key }&	 DES key is a weak key \\
{\sc krb5_bad_keytype }&	 Keytype is incompatible with encryption type \\
{\sc krb5_bad_keysize }&	 Key size is incompatible with encryption type \\
{\sc krb5_bad_msize }&	 Message size is incompatible with encryption type \\
{\sc krb5_cc_type_exists }&	 Credentials cache type is already registered. \\
{\sc krb5_kt_type_exists }&	 Key table type is already registered. \\
{\sc krb5_cc_io }&		 Credentials cache I/O operation failed XXX \\
{\sc krb5_fcc_perm }&	 Credentials cache file permissions incorrect \\
{\sc krb5_fcc_nofile }&	 No credentials cache file found \\
{\sc krb5_fcc_internal }&	 Internal file credentials cache error \\
{\sc krb5_cc_nomem }& \parbox[t]{3in}{No more memory to allocate (in credentials cache code)}\\ 
\end{tabular}

\begin{tabular}{ll}
\multicolumn{2}{c}{errors for dual TGT library calls} \\

{\sc krb5_invalid_flags }& Invalid KDC option combination (library internal error) \\
{\sc krb5_no_2nd_tkt }&	 Request missing second ticket \\
{\sc krb5_nocreds_supplied }& No credentials supplied to library routine \\

\end{tabular}

\begin{tabular}{ll}
\multicolumn{2}{c}{errors for sendauth and recvauth} \\

{\sc krb5_sendauth_badauthvers }& Bad sendauth version was sent \\
{\sc krb5_sendauth_badapplvers }& Bad application version was sent (via sendauth) \\
{\sc krb5_sendauth_badresponse }& Bad response (during sendauth exchange) \\
{\sc krb5_sendauth_rejected }& Server rejected authentication\\
& \ (during sendauth exchange) \\
{\sc krb5_sendauth_mutual_failed }& Mutual authentication failed\\&\ (during sendauth exchange) \\

\end{tabular}

\begin{tabular}{ll}
\multicolumn{2}{c}{errors for preauthentication} \\

{\sc krb5_preauth_bad_type }& Unsupported preauthentication type \\
{\sc krb5_preauth_no_key }&	 Required preauthentication key not supplied \\
{\sc krb5_preauth_failed }&	 Generic preauthentication failure \\

\end{tabular}

\begin{tabular}{ll}
\multicolumn{2}{c}{version number errors} \\

{\sc krb5_rcache_badvno }& Unsupported replay cache format version number \\
{\sc krb5_ccache_badvno }& Unsupported credentials cache format version number \\
{\sc krb5_keytab_badvno }& Unsupported key table format version number \\

\end{tabular}

\begin{tabular}{ll}
\multicolumn{2}{c}{other errors} \\ 

{\sc krb5_prog_atype_nosupp }& Program lacks support for address type \\
{\sc krb5_rc_required }& Message replay detection requires\\&\  rcache parameter \\
{\sc krb5_err_bad_hostname }& Hostname cannot be canonicalized \\
{\sc krb5_err_host_realm_unknown }& Cannot determine realm for host \\
{\sc krb5_sname_unsupp_nametype }& Conversion to service principal undefined\\&\ for name type \\
{\sc krb5krb_ap_err_v4_reply }& Initial Ticket Response appears to be\\
&\ Version 4 error \\
{\sc krb5_realm_cant_resolve }& Cannot resolve KDC for requested realm \\
{\sc krb5_tkt_not_forwardable }& Requesting ticket can't get forwardable tickets \\
\end{tabular}
\end{small}

\subsection{error_table kdb5}

% $Source$
% $Author$

The Kerberos v5 database library error code table

\begin{small}
\begin{tabular}{ll}
\multicolumn{2}{c}{From the server side routines} \\
{\sc krb5_kdb_inuse }&	Entry already exists in database\\
{\sc krb5_kdb_uk_serror }&	Database store error\\
{\sc krb5_kdb_uk_rerror }&	Database read error\\
{\sc krb5_kdb_unauth }&	Insufficient access to perform requested operation\\
{\sc krb5_kdb_noentry }&	No such entry in the database\\
{\sc krb5_kdb_ill_wildcard }& Illegal use of wildcard\\
{\sc krb5_kdb_db_inuse }&	Database is locked or in use--try again later\\
{\sc krb5_kdb_db_changed }&	Database was modified during read\\
{\sc krb5_kdb_truncated_record }&	Database record is incomplete or corrupted\\
{\sc krb5_kdb_recursivelock }&	Attempt to lock database twice\\
{\sc krb5_kdb_notlocked }&		Attempt to unlock database when not locked\\
{\sc krb5_kdb_badlockmode }&	Invalid kdb lock mode\\
{\sc krb5_kdb_dbnotinited }&	Database has not been initialized\\
{\sc krb5_kdb_dbinited }&		Database has already been initialized\\
{\sc krb5_kdb_illdirection }&	Bad direction for converting keys\\
{\sc krb5_kdb_nomasterkey }&	Cannot find master key record in database\\
{\sc krb5_kdb_badmasterkey }&	Master key does not match database\\
{\sc krb5_kdb_invalidkeysize }&	Key size in database is invalid\\
{\sc krb5_kdb_cantread_stored }&	Cannot find/read stored master key\\
{\sc krb5_kdb_badstored_mkey }&	Stored master key is corrupted\\
{\sc krb5_kdb_cantlock_db }&	Insufficient access to lock database \\
{\sc krb5_kdb_db_corrupt }&		Database format error\\
{\sc krb5_kdb_bad_version }&	Unsupported version in database entry \\
\end{tabular}
\end{small}

% $Source$
% $Author$

\subsection{The Kerberos v5 magic numbers errorcode table}

\begin{small}
\begin{tabular}{ll} 
{\sc kv5m_none }&		Kerberos V5 magic number table \\
{\sc kv5m_principal }&	Bad magic number for krb5_principal structure \\
{\sc kv5m_data }&		Bad magic number for krb5_data structure \\
{\sc kv5m_keyblock }&	Bad magic number for krb5_keyblock structure \\
{\sc kv5m_checksum }&	Bad magic number for krb5_checksum structure \\
{\sc kv5m_encrypt_block }&	Bad magic number for krb5_encrypt_block structure \\
{\sc kv5m_enc_data }&	Bad magic number for krb5_enc_data structure \\
{\sc kv5m_cryptosystem_entry }&	Bad magic number for krb5_cryptosystem_entry\\&\ structure \\
{\sc kv5m_cs_table_entry }&	Bad magic number for krb5_cs_table_entry structure \\
{\sc kv5m_checksum_entry }&	Bad magic number for krb5_checksum_entry structure \\

{\sc kv5m_authdata }&	Bad magic number for krb5_authdata structure \\
{\sc kv5m_transited }&	Bad magic number for krb5_transited structure \\
{\sc kv5m_enc_tkt_parT }&	Bad magic number for krb5_enc_tkt_part structure \\
{\sc kv5m_ticket }&		Bad magic number for krb5_ticket structure \\
{\sc kv5m_authenticator }&	Bad magic number for krb5_authenticator structure \\
{\sc kv5m_tkt_authent }&	Bad magic number for krb5_tkt_authent structure \\
{\sc kv5m_creds }&		Bad magic number for krb5_creds structure \\
{\sc kv5m_last_req_entry }&	Bad magic number for krb5_last_req_entry structure \\
{\sc kv5m_pa_data }&		Bad magic number for krb5_pa_data structure \\
{\sc kv5m_kdc_req }&		Bad magic number for krb5_kdc_req structure \\
{\sc kv5m_enc_kdc_rep_part }& Bad magic number for krb5_enc_kdc_rep_part structure \\
{\sc kv5m_kdc_rep }&		Bad magic number for krb5_kdc_rep structure \\
{\sc kv5m_error }&		Bad magic number for krb5_error structure \\
{\sc kv5m_ap_req }&		Bad magic number for krb5_ap_req structure \\
{\sc kv5m_ap_rep }&		Bad magic number for krb5_ap_rep structure \\
{\sc kv5m_ap_rep_enc_part }&	Bad magic number for krb5_ap_rep_enc_part structure \\
{\sc kv5m_response }&	Bad magic number for krb5_response structure \\
{\sc kv5m_safe }&		Bad magic number for krb5_safe structure \\
{\sc kv5m_priv }&		Bad magic number for krb5_priv structure \\
{\sc kv5m_priv_enc_part }&	Bad magic number for krb5_priv_enc_part structure \\
{\sc kv5m_cred }&		Bad magic number for krb5_cred structure \\
{\sc kv5m_cred_info }&	Bad magic number for krb5_cred_info structure \\
{\sc kv5m_cred_enc_part }&	Bad magic number for krb5_cred_enc_part structure \\
{\sc kv5m_pwd_data }&	Bad magic number for krb5_pwd_data structure \\
{\sc kv5m_address }&	Bad magic number for krb5_address structure \\
{\sc kv5m_keytab_entry }&	Bad magic number for krb5_keytab_entry structure \\
{\sc kv5m_context }&	Bad magic number for krb5_context structure \\
{\sc kv5m_os_context }&	Bad magic number for krb5_os_context structure \\

\end{tabular}
\end{small}

\subsection{error_table asn1}

The Kerberos v5/ASN.1 error table mappings

\begin{small}
\begin{tabular}{ll}
{\sc asn1_bad_timeformat }&	ASN.1 failed call to system time library \\
{\sc asn1_missing_field }&	ASN.1 structure is missing a required field \\
{\sc asn1_misplaced_field }&	ASN.1 unexpected field number \\
{\sc asn1_type_mismatch }&	ASN.1 type numbers are inconsistent \\
{\sc asn1_overflow }&	ASN.1 value too large \\
{\sc asn1_overrun }&	ASN.1 encoding ended unexpectedly \\
{\sc asn1_bad_id }&	ASN.1 identifier doesn't match expected value \\
{\sc asn1_bad_length }&	ASN.1 length doesn't match expected value \\
{\sc asn1_bad_format }&	ASN.1 badly-formatted encoding \\
{\sc asn1_parse_error }&	ASN.1 parse error \\
\end{tabular}
\end{small}



%\addtolength{\oddsidemargin}{-1in}
%\addtolength{\evensidemargin}{1.00in}
%\addtolength{\textwidth}{-1.75in}
\newpage

\section{libkrb5.a functions}
This section describes the functions provided in the \libname{libkrb5.a}
library.  The library is built from several pieces, mostly for convenience in
programming, maintenance, and porting.

\ifdraft\sloppy\fi

\subsection{Main functions}
The main functions deal with the nitty-gritty details: verifying
tickets, creating authenticators, and the like.

\begin{funcdecl}[krb5_encode_kdc_rep]{krb5_error_code}{\funcin}
\funcarg{krb5_msgtype}{type}
\funcarg{krb5_enc_kdc_rep_part *}{encpart}
\funcarg{krb5_keyblock *}{client_key}
\funcinout
\funcarg{krb5_kdc_rep *}{dec_rep}
\funcout
\funcarg{krb5_data *}{enc_rep}
\end{funcdecl}

Takes KDC rep parts in \funcparam{*rep} and \funcparam{*encpart}, and
formats it into \funcparam{*enc_rep}, using message type \funcparam{type}
and encryption key \funcparam{client_key} and encryption type
\funcparam{dec_rep{\ptsto}etype}.

\funcparam{enc_rep{\ptsto}data} will point to  allocated storage upon
non-error return; the caller should free it when finished.

Returns system errors.

\begin{funcdecl}[krb5_decode_kdc_rep]{krb5_error_code}{\funcin}
\funcarg{krb5_data *}{enc_rep}
\funcarg{krb5_keyblock *}{key}
\funcarg{krb5_enctype}{etype}
\funcout
\funcarg{krb5_kdc_rep **}{dec_rep}
\end{funcdecl}

Takes a KDC_REP message and decrypts encrypted part using
\funcparam{etype} and \funcparam{*key}, putting result in \funcparam{*rep}.
The pointers in \funcparam{dec_rep}
are all set to allocated storage which should be freed by the caller
when finished with the response (by using \funcname{krb5_free_kdc_rep}).


If the response isn't a KDC_REP (tgs or as), it returns an error from
the decoding routines (usually ISODE_50_LOCAL_ERR_BADDECODE).

Returns errors from encryption routines, system errors.



\subsection{Credentials cache functions}
The credentials cache functions (some of which are macros which call to
specific types of credentials caches) deal with storing credentials
(tickets, session keys, and other identifying information) in a
semi-permanent store for later use by different programs.

\subsubsection{Per-type functions}
The following entry points must be implemented for each type of
credentials cache; however, applications are not expected to have a need
to call either \funcname{krb5_cc_resolve_internal} or
\funcname{krb5_cc_gennew_internal}.


\begin{funcdecl}{krb5_error_code}{krb5_cc_resolve_internal}{\funcout}
\funcarg{krb5_ccache *}{id}
\funcin
\funcarg{char *}{residual}
\end{funcdecl}

Creates a credentials cache named by \funcparam{residual} (which may be
interpreted differently by each type of ccache).  The cache is not
opened, but the cache name is held in reserve.

\begin{funcdecl}{krb5_error_code}{krb5_cc_gennew_internal}{\funcout}
\funcarg{krb5_ccache *}{id}
\end{funcdecl}

Creates a new credentials cache whose name is guaranteed to be
unique.  The cache is not opened. \funcparam{*id} is
filled in with a \datatype{krb5_ccache} which may be used in subsequent
calls to ccache functions.

\begin{funcdecl}{krb5_cc_initialize}{krb5_error_code}{\funcinout}
\funcarg{krb5_ccache}{id}
\funcin
\funcarg{krb5_principal}{primary_principal}
\end{funcdecl}

Creates/refreshes a credentials cache identified by \funcparam{id} with
primary principal set to \funcparam{primary_principal}.
If the credentials cache already exists, its contents are destroyed.

Errors:  permission errors, system errors.

Modifies: cache identified by \funcparam{id}.

\begin{funcdecl}{krb5_cc_destroy}{krb5_error_code}{\funcin}
\funcarg{krb5_ccache}{id}
\end{funcdecl}

Destroys the credentials cache identified by \funcparam{id}.
Requires that the credentials cache exist.

Errors:  permission errors.

\begin{funcdecl}{krb5_cc_close}{krb5_error_code}{\funcinout}
\funcarg{krb5_ccache}{id}
\end{funcdecl}

Closes the credentials cache \funcparam{id}, invalidates \funcparam{id},
and releases any other resources acquired during use of the credentials cache.
Requires that \funcparam{id} identify a valid credentials cache.


\begin{funcdecl}{krb5_cc_store_cred}{krb5_error_code}{\funcin}
\funcarg{krb5_ccache}{id}
\funcarg{krb5_credentials *}{creds}
\end{funcdecl}

Stores \funcparam{creds} in the cache \funcparam{id}, tagged with
\funcparam{creds{\ptsto}client}.
Requires that \funcparam{id} identify a valid credentials cache.

Errors: permission errors, storage failure errors.

\begin{funcdecl}{krb5_cc_retrieve_cred}{krb5_error_code}{\funcin}
\funcarg{krb5_ccache}{id}
\funcarg{krb5_flags}{whichfields}
\funcarg{krb5_credentials *}{mcreds}
\funcout
\funcarg{krb5_credentials *}{creds}
\end{funcdecl}

Searches the cache \funcparam{id} for credentials matching
\funcparam{mcreds}.  The fields which are to be matched are specified by
set bits in \funcparam{whichfields}, and always include the principal
name \funcparam{mcreds{\ptsto}server}.
Requires that \funcparam{id} identify a valid credentials cache.

If at least one match is found, one of the matching credentials is
returned in \funcparam{*creds}. XXX free the return creds?

Errors: error code if no matches found.

\begin{funcdecl}{krb5_cc_get_principal}{krb5_error_code}{\funcin}
\funcarg{krb5_ccache}{id}
\funcarg{krb5_principal *}{principal}
\end{funcdecl}

Retrieves the primary principal of the credentials cache (as
set by the \funcname{krb5_cc_initialize} request)
The primary principal is filled into \funcparam{*principal}; the caller
should release this memory by calling \funcname{krb5_free_principal} on
\funcparam{*principal} when finished.

Requires that \funcparam{id} identify a valid credentials cache.

\begin{funcdecl}{krb5_cc_start_seq_get}{krb5_error_code}{\funcin}
\funcarg{krb5_ccache}{id}
\funcout
\funcarg{krb5_cc_cursor *}{cursor}
\end{funcdecl}

Prepares to sequentially read every set of cached credentials.
Requires that \funcparam{id} identify a valid credentials cache opened by
\funcname{krb5_cc_open}.
\funcparam{cursor} is filled in with a cursor to be used in calls to
\funcname{krb5_cc_next_cred}.

\begin{funcdecl}{krb5_cc_next_cred}{krb5_error_code}{\funcin}
\funcarg{krb5_ccache}{id}
\funcout
\funcarg{krb5_credentials *}{creds}
\funcinout
\funcarg{krb5_cc_cursor *}{cursor}
\end{funcdecl}

Fetches the next entry from \funcparam{id}, returning its values in
\funcparam{*creds}, and updates \funcparam{*cursor} for the next request.
Requires that \funcparam{id} identify a valid credentials cache and
\funcparam{*cursor} be a cursor returned by
\funcname{krb5_cc_start_seq_get} or a subsequent call to
\funcname{krb5_cc_next_cred}.

Errors: error code if no more cache entries.

\begin{funcdecl}{krb5_cc_end_seq_get}{krb5_error_code}{\funcin}
\funcarg{krb5_ccache}{id}
\funcarg{krb5_cc_cursor *}{cursor}
\end{funcdecl}

Finishes sequential processing mode and invalidates \funcparam{*cursor}.
\funcparam{*cursor} must never be re-used after this call.

Requires that \funcparam{id} identify a valid credentials cache and
\funcparam{*cursor} be a cursor returned by
\funcname{krb5_cc_start_seq_get} or a subsequent call to
\funcname{krb5_cc_next_cred}.

Errors: may return error code if \funcparam{*cursor} is invalid.


\begin{funcdecl}{krb5_cc_remove_cred}{krb5_error_code}{\funcin}
\funcarg{krb5_ccache}{id}
\funcarg{krb5_flags}{which}
\funcarg{krb5_credentials *}{cred}
\end{funcdecl}

Removes any credentials from \funcparam{id} which match the principal
name {cred{\ptsto}server} and the fields in \funcparam{cred} masked by
\funcparam{which}.
Requires that \funcparam{id} identify a valid credentials cache.

Errors: returns error code if nothing matches; returns error code if
couldn't delete.

\begin{funcdecl}{krb5_cc_set_flags}{krb5_error_code}{\funcin}
\funcarg{krb5_ccache}{id}
\funcarg{krb5_flags}{flags}
\end{funcdecl}

Sets the flags on the cache \funcparam{id} to \funcparam{flags}.


\subsubsection{Glue functions}
The following functions are implemented in the base library and serve to
glue together the various types of credentials caches.


\begin{funcdecl}{krb5_cc_resolve}{krb5_error_code}{\funcin}
\funcarg{char *}{string_name}
\funcout
\funcarg{krb5_ccache *}{id}
\end{funcdecl}

Fills in \funcparam{id} with a ccache identifier which corresponds to
the name in \funcparam{string_name}.  The cache is left unopened.

Requires that \funcparam{string_name} be of the form ``type:residual'' and
``type'' is a type known to the library.

\begin{funcdecl}{krb5_cc_generate_new}{krb5_error_code}{\funcin}
\funcarg{krb5_cc_ops *}{ops}
\funcout
\funcarg{krb5_ccache *}{id}
\end{funcdecl}


Fills in \funcparam{id} with a unique ccache identifier of a type defined by
\funcparam{ops}.  The cache is left unopened.

\begin{funcdecl}{krb5_cc_register}{krb5_error_code}{\funcin}
\funcarg{krb5_cc_ops *}{ops}
\funcarg{krb5_boolean}{override}
\end{funcdecl}

Adds a new cache type identified and implemented by \funcparam{ops} to
the set recognized by \funcname{krb5_cc_resolve}.
If \funcparam{override} is FALSE, a ticket cache type named
\funcparam{ops{\ptsto}prefix} must not be known.

\begin{funcdecl}{krb5_cc_get_name}{char *}{\funcin}
\funcarg{krb5_ccache}{id}
\end{funcdecl}

Returns the name of the ccache denoted by \funcparam{id}.

\begin{funcdecl}{krb5_cc_default_name}{char *}{\funcvoid}
\end{funcdecl}

Returns the name of the default credentials cache; this may be equivalent to
{\funcfont getenv}({\tt "KRB5CCACHE"}) with an appropriate fallback.

\begin{funcdecl}{krb5_cc_default }{krb5_error_code}{\funcout}
\funcarg{krb5_ccache *}{ccache}
\end{funcdecl}

Equivalent to {\funcfont krb5_cc_resolve}({\funcfont
krb5_cc_default_name}(), \funcparam{ccache}).



\subsection{Replay cache functions}
The replay cache functions deal with verifying that AP_REQ's do not
contain duplicate authenticators; the storage must be non-volatile for
the site-determined validity period of authenticators.

Each replay cache has a string ``name'' associated with it.  The use of
this name is dependent on the underlying caching strategy (for
file-based things, it would be a cache file name).  The
caching strategy should use non-volatile storage so that replay
integrity can be maintained across system failures.

\subsubsection{Per-type functions}
The following entry points must be implemented for each type of
credentials cache.

\begin{funcdecl}{krb5_rc_initialize}{krb5_error_code}{\funcin}
\funcarg{krb5_rcache}{id}
\funcarg{krb5_deltat}{auth_lifespan}
\end{funcdecl}

Creates/refreshes the replay cache identified by \funcparam{id} and sets its
authenticator lifespan to \funcparam{auth_lifespan}.  If the 
replay cache already exists, its contents are destroyed.

Errors: permission errors, system errors

\begin{funcdecl}{krb5_rc_recover}{krb5_error_code}{\funcin}
\funcarg{krb5_rcache}{id}
\end{funcdecl}
Attempts to recover the replay cache \funcparam{id}, (presumably after a
system crash or server restart).

Errors: error indicating that no cache was found to recover

\begin{funcdecl}{krb5_rc_destroy}{krb5_error_code}{\funcin}
\funcarg{krb5_rcache}{id}
\end{funcdecl}

Destroys the replay cache \funcparam{id}.
Requires that \funcparam{id} identifies a valid replay cache.

Errors: permission errors.

\begin{funcdecl}{krb5_rc_close}{krb5_error_code}{\funcin}
\funcarg{krb5_rcache}{id}
\end{funcdecl}

Closes the replay cache \funcparam{id}, invalidates \funcparam{id},
and releases any other resources acquired during use of the replay cache.
Requires that \funcparam{id} identifies a valid replay cache.

Errors: permission errors

\begin{funcdecl}{krb5_rc_store}{krb5_error_code}{\funcin}
\funcarg{krb5_rcache}{id}
\funcarg{krb5_dont_replay *}{rep}
\end{funcdecl}
Stores \funcparam{rep} in the replay cache \funcparam{id}.
Requires that \funcparam{id} identifies a valid replay cache.

Returns KRB5KRB_AP_ERR_REPEAT if \funcparam{rep} is already in the
cache.  May also return permission errors, storage failure errors.

\begin{funcdecl}{krb5_rc_expunge}{krb5_error_code}{\funcin}
\funcarg{krb5_rcache}{id}
\end{funcdecl}
Removes all expired replay information (i.e. those entries which are
older than then authenticator lifespan of the cache) from the cache
\funcparam{id}.  Requires that \funcparam{id} identifies a valid replay
cache.

Errors: permission errors.

\begin{funcdecl}{krb5_rc_get_lifespan}{krb5_error_code}{\funcin}
\funcarg{krb5_rcache}{id}
\funcout
\funcarg{krb5_deltat *}{auth_lifespan}
\end{funcdecl}
Fills in \funcparam{auth_lifespan} with the lifespan of
the cache \funcparam{id}.
Requires that \funcparam{id} identifies a valid replay cache.

\begin{funcdecl}{krb5_rc_resolve}{krb5_error_code}{\funcinout}
\funcarg{krb5_rcache}{id}
\funcin
\funcarg{char *}{name}
\end{funcdecl}

Initializes private data attached to \funcparam{id}.  This function MUST
be called before the other per-replay cache functions.

Requires that \funcparam{id} points to allocated space, with an
initialized \funcparam{id{\ptsto}ops} field.

Returns:  allocation errors.


\begin{funcdecl}{krb5_rc_get_name}{char *}{\funcin}
\funcarg{krb5_rcache}{id}
\end{funcdecl}

Returns the name (excluding the type) of the rcache \funcparam{id}.
Requires that \funcparam{id} identifies a valid replay cache.

\subsubsection{Glue functions}
The following functions are implemented in the base library and serve to
glue together the various types of replay caches.

\begin{funcdecl}{krb5_rc_resolve_full}{krb5_error_code}{\funcinout}
\funcarg{krb5_rcache *}{id}
\funcin
\funcarg{char *}{string_name}
\end{funcdecl}

\funcparam{id} is filled in to identify a replay cache which
corresponds to the name in \funcparam{string_name}.  The cache is not opened.
Requires that \funcparam{string_name} be of the form ``type:residual''
and that ``type'' is a type known to the library.

Errors: error if cannot resolve name.

\begin{funcdecl}{krb5_rc_register_type}{krb5_error_code}{\funcin}
\funcarg{krb5_rc_ops *}{ops}
\end{funcdecl}
Adds a new replay cache type implemented and identified by
\funcparam{ops} to the set recognized by
\funcname{krb5_rc_resolve}.  Requires that a ticket cache type named
\funcparam{ops{\ptsto}prefix} is not yet known.


\begin{funcdecl}{krb5_rc_default_name}{char *}{\funcvoid}
\end{funcdecl}
Returns  the name of the default replay cache; this may be equivalent to
\funcnamenoparens{getenv}({\tt "KRB5RCACHE"}) with an appropriate fallback.

\begin{funcdecl}{krb5_rc_default_type}{char *}{\funcvoid}
\end{funcdecl}

Returns the type of the default replay cache.

\begin{funcdecl}{krb5_rc_default}{krb5_error_code}{\funcinout}
\funcarg{krb5_rcache *}{id}
\end{funcdecl}
Equivalent to \funcnamenoparens{krb5_rc_resolve_full}(\funcparam{id},
\funcnamenoparens{strcat}(\funcnamenoparens{strcat}(\funcname{krb5_rc_default_type},``:''),
\funcname{krb5_rc_default_name})) (except of course you can't do the
strcat's with the return values\ldots).


\subsection{Key table functions}
The key table functions deal with storing and retrieving service keys
for use by unattended services which participate in authentication exchanges.

Keytab routines are all be atomic.  Every routine that acquires
a non-sharable resource releases it before it returns. 

All keytab types support multiple concurrent sequential scans.

The order of values returned from \funcname{krb5_kt_next_entry} is
unspecified.

Although the ``right thing'' should happen if the program aborts
abnormally, a close routine, \funcname{krb5_kt_free_entry},  is provided
for freeing resources, etc.  People should use the close routine when
they are finished.

\begin{funcdecl}{krb5_kt_register}{krb5_error_code}{\funcinout}
\funcarg{krb5_context}{context}
\funcin
\funcarg{krb5_kt_ops *}{ops}
\end{funcdecl}


Adds a new ticket cache type to the set recognized by
\funcname{krb5_kt_resolve}.
Requires that a keytab type named \funcparam{ops{\ptsto}prefix} is not
yet known.

An error is returned if \funcparam{ops{\ptsto}prefix} is already known.

\begin{funcdecl}{krb5_kt_resolve}{krb5_error_code}{\funcinout}
\funcarg{krb5_context}{context}
\funcin
\funcarg{const char *}{string_name}
\funcout
\funcarg{krb5_keytab *}{id}
\end{funcdecl}

Fills in \funcparam{*id} with a handle identifying the keytab with name
``string_name''.  The keytab is not opened.
Requires that \funcparam{string_name} be of the form ``type:residual'' and
``type'' is a type known to the library.

Errors: badly formatted name.
		
\begin{funcdecl}{krb5_kt_default_name}{krb5_error_code}{\funcinout}
\funcarg{krb5_context}{context}
\funcin
\funcarg{char *}{name}
\funcarg{int}{namesize}
\end{funcdecl}

\funcparam{name} is filled in with the first \funcparam{namesize} bytes of
the name of the default keytab.
If the name is shorter than \funcparam{namesize}, then the remainder of
\funcparam{name} will be zeroed.


\begin{funcdecl}{krb5_kt_default}{krb5_error_code}{\funcinout}
\funcarg{krb5_context}{context}
\funcin
\funcarg{krb5_keytab *}{id}
\end{funcdecl}

Fills in \funcparam{id} with a handle identifying the default keytab.

\begin{funcdecl}{krb5_kt_read_service_key}{krb5_error_code}{\funcinout}
\funcarg{krb5_context}{context}
\funcin
\funcarg{krb5_pointer}{keyprocarg}
\funcarg{krb5_principal}{principal}
\funcarg{krb5_kvno}{vno}
\funcarg{krb5_keytype}{keytype}
\funcout
\funcarg{krb5_keyblock **}{key}
\end{funcdecl}

If \funcname{keyprocarg} is not NULL, it is taken to be a
\datatype{char *} denoting the name of a keytab.  Otherwise, the default
keytab will be used.
The keytab is opened and searched for the entry identified by
\funcparam{principal}, \funcparam{keytype}, and \funcparam{vno}, 
returning the resulting key
in \funcparam{*key} or returning an error code if it is not found. 

\funcname{krb5_free_keyblock} should be called on \funcparam{*key} when
the caller is finished with the key.

Returns an error code if the entry is not found.

\begin{funcdecl}{krb5_kt_add_entry}{krb5_error_code}{\funcinout}
\funcarg{krb5_context}{context}
\funcin
\funcarg{krb5_keytab}{id}
\funcarg{krb5_keytab_entry *}{entry}
\end{funcdecl}

Calls the keytab-specific add routine \funcname{krb5_kt_add_internal}
with the same function arguments.  If this routine is not available,
then KRB5_KT_NOWRITE is returned.

\begin{funcdecl}{krb5_kt_remove_entry}{krb5_error_code}{\funcinout}
\funcarg{krb5_context}{context}
\funcin
\funcarg{krb5_keytab}{id}
\funcarg{krb5_keytab_entry *}{entry}
\end{funcdecl}

Calls the keytab-specific remove routine
\funcname{krb5_kt_remove_internal} with the same function arguments.
If this routine is not available, then KRB5_KT_NOWRITE is returned.

\begin{funcdecl}{krb5_kt_get_name}{krb5_error_code}{\funcinout}
\funcarg{krb5_context}{context}
\funcarg{krb5_keytab}{id}
\funcout
\funcarg{char *}{name}
\funcin
\funcarg{int}{namesize}
\end{funcdecl}

\funcarg{name} is filled in with the first \funcparam{namesize} bytes of
the name of the keytab identified by \funcname{id}.
If the name is shorter than \funcparam{namesize}, then \funcarg{name}
will be null-terminated.

\begin{funcdecl}{krb5_kt_close}{krb5_error_code}{\funcinout}
\funcarg{krb5_context}{context}
\funcarg{krb5_keytab}{id}
\end{funcdecl}

Closes the keytab identified by \funcparam{id} and invalidates
\funcparam{id}, and releases any other resources acquired during use of
the key table.

Requires that \funcparam{id} identifies a keytab.

\begin{funcdecl}{krb5_kt_get_entry}{krb5_error_code}{\funcinout}
\funcarg{krb5_context}{context}
\funcarg{krb5_keytab}{id}
\funcin
\funcarg{krb5_principal}{principal}
\funcarg{krb5_kvno}{vno}
\funcarg{krb5_keytype}{keytype}
\funcout
\funcarg{krb5_keytab_entry *}{entry}
\end{funcdecl}

\begin{sloppypar}
Searches the keytab identified by \funcparam{id} for an entry whose
principal matches \funcparam{principal}, whose keytype matches 
\funcparam{keytype}, and
whose key version number matches \funcparam{vno}.  If \funcparam{vno} is
zero, the first entry whose principal matches is returned.
\end{sloppypar}

Returns an error code if no suitable entry is found.  If an entry is
found, the entry is returned in \funcparam{*entry}; its contents should
be deallocated by calling \funcname{krb5_kt_free_entry} when no longer
needed.

\begin{funcdecl}{krb5_kt_free_entry}{krb5_error_code}{\funcinout}
\funcarg{krb5_context}{context}
\funcarg{krb5_keytab_entry *}{entry}
\end{funcdecl}

Releases all storage allocated for \funcparam{entry}, which must point
to a structure previously filled in by \funcname{krb5_kt_get_entry} or
\funcname{krb5_kt_next_entry}.

\begin{funcdecl}{krb5_kt_start_seq_get}{krb5_error_code}{\funcinout}
\funcarg{krb5_context}{context}
\funcarg{krb5_keytab}{id}
\funcout
\funcarg{krb5_kt_cursor *}{cursor}
\end{funcdecl}

Prepares to read sequentially every key in the keytab identified by
\funcparam{id}.
\funcparam{cursor} is filled in with a cursor to be used in calls to
\funcname{krb5_kt_next_entry}.

\begin{funcdecl}{krb5_kt_next_entry}{krb5_error_code}{\funcinout}
\funcarg{krb5_context}{context}
\funcarg{krb5_keytab}{id}
\funcout
\funcarg{krb5_keytab_entry *}{entry}
\funcinout
\funcarg{krb5_kt_cursor}{cursor}
\end{funcdecl}

Fetches the ``next'' entry in the keytab, returning it in
\funcparam{*entry}, and updates \funcparam{*cursor} for the next
request.  If the keytab changes during the sequential get, an error is
guaranteed.  \funcparam{*entry} should be freed after use by calling
\funcname{krb5_kt_free_entry}.

Requires that \funcparam{id} identifies a valid keytab.  and
\funcparam{*cursor} be a cursor returned by
\funcname{krb5_kt_start_seq_get} or a subsequent call to
\funcname{krb5_kt_next_entry}.

Errors: error code if no more cache entries or if the keytab changes.

\begin{funcdecl}{krb5_kt_end_seq_get}{krb5_error_code}{\funcinout}
\funcarg{krb5_context}{context}
\funcarg{krb5_keytab}{id}
\funcarg{krb5_kt_cursor *}{cursor}
\end{funcdecl}

Finishes sequential processing mode and invalidates \funcparam{cursor},
which must never be re-used after this call.

Requires that \funcparam{id} identifies a valid keytab  and
\funcparam{*cursor} be a cursor returned by
\funcname{krb5_kt_start_seq_get} or a subsequent call to
\funcname{krb5_kt_next_entry}.

May return error code if \funcparam{cursor} is invalid.




\subsection{Free functions}
The free functions deal with deallocation of memory that has been
allocated by various routines. It is recommended that the developer use
these routines as they will know about the contents of the structures.

\begin{funcdecl}{krb5_auth_con_free}{krb5_auth_con_free}{\funcinout}
\funcarg{krb5_context}{context}
\funcarg{krb5_auth_context *}{auth_context}
\end{funcdecl}

Frees the auth_context \funcparam{auth_context} returned by
\funcname{krb5_auth_con_init}.

\begin{funcdecl}{krb5_free_context}{void}{\funcinout}
\funcarg{krb5_context}{context}
\end{funcdecl}

Frees the context returned by \funcname{krb5_init_context}. Internally
calls \funcname{krb5_os_free_context}.

\begin{funcdecl}{krb5_free_princial}{void}{\funcinout}
\funcarg{krb5_context}{context}
\funcarg{krb5_principal}{val}
\end{funcdecl}

Frees the pwd_data \funcparam{val} that has been allocated from
\funcname{krb5_copy_principal}. 

\begin{funcdecl}{krb5_free_authenticator}{void}{\funcinout}
\funcarg{krb5_context}{context}
\funcarg{krb5_authenticator *}{val}
\end{funcdecl}

Frees the authenticator \funcparam{val}, including the pointer
\funcparam{val}. 

\begin{funcdecl}{krb5_free_authenticator_contents}{void}{\funcinout}
\funcarg{krb5_context}{context}
\funcarg{krb5_authenticator *}{val}
\end{funcdecl}

Frees the authenticator contents of \funcparam{val}. The pointer 
\funcparam{val} is not freed.


\begin{funcdecl}{krb5_free_addresses}{void}{\funcinout}
\funcarg{krb5_context}{context}
\funcarg{krb5_address **}{val}
\end{funcdecl}

Frees the series of addresses \funcparam{*val} that have been allocated from
\funcname{krb5_copy_addresses}. 

\begin{funcdecl}{krb5_free_address}{void}{\funcinout}
\funcarg{krb5_context}{context}
\funcarg{krb5_address *}{val}
\end{funcdecl}

Frees the address \funcparam{val}.

\begin{funcdecl}{krb5_free_authdata}{void}{\funcinout}
\funcarg{krb5_context}{context}
\funcarg{krb5_authdata **}{val}
\end{funcdecl}

Frees the authdata structure pointed to by \funcparam{val} that has been
allocated from 
\funcname{krb5_copy_authdata}. 

\begin{funcdecl}{krb5_free_enc_tkt_part}{void}{\funcinout}
\funcarg{krb5_context}{context}
\funcarg{krb5_enc_tkt_part *}{val}
\end{funcdecl}

Frees \funcparam{val} that has been allocated from
\funcname{krb5_enc_tkt_part} and \funcname{krb5_decrypt_tkt_part}.

\begin{funcdecl}{krb5_free_ticket}{void}{\funcinout}
\funcarg{krb5_context}{context}
\funcarg{krb5_ticket *}{val}
\end{funcdecl}

Frees the ticket \funcparam{val} that has been allocated from
\funcname{krb5_copy_ticket} and other routines.

\begin{funcdecl}{krb5_free_tickets}{void}{\funcinout}
\funcarg{krb5_context}{context}
\funcarg{krb5_ticket **}{val}
\end{funcdecl}

Frees the tickets pointed to by \funcparam{val}.

\begin{funcdecl}{krb5_free_kdc_req}{void}{\funcinout}
\funcarg{krb5_context}{context}
\funcarg{krb5_kdc_req *}{val}
\end{funcdecl}

Frees the kdc_req \funcparam{val} and all substructures. The pointer
\funcparam{val} is freed as well.

\begin{funcdecl}{krb5_free_kdc_rep}{void}{\funcinout}
\funcarg{krb5_context}{context}
\funcarg{krb5_kdc_rep *}{val}
\end{funcdecl}

Frees the kdc_rep \funcparam{val} that has been allocated from
\funcname{krb5_get_in_tkt}. 

\begin{funcdecl}{krb5_free_kdc_rep_part}{void}{\funcinout}
\funcarg{krb5_context}{context}
\funcarg{krb5_enc_kdc_rep_part *}{val}
\end{funcdecl}

Frees the kdc_rep_part \funcparam{val}.

\begin{funcdecl}{krb5_free_error}{void}{\funcinout}
\funcarg{krb5_context}{context}
\funcarg{krb5_error *}{val}
\end{funcdecl}

Frees the error \funcparam{val} that has been allocated from
\funcname{krb5_read_error} or \funcname{krb5_sendauth}. 

\begin{funcdecl}{krb5_free_ap_req}{void}{\funcinout}
\funcarg{krb5_context}{context}
\funcarg{krb5_ap_req *}{val}
\end{funcdecl}

Frees the ap_req \funcparam{val}.

\begin{funcdecl}{krb5_free_ap_rep}{void}{\funcinout}
\funcarg{krb5_context}{context}
\funcarg{krb5_ap_rep *}{val}
\end{funcdecl}

Frees the ap_rep \funcparam{val}.

\begin{funcdecl}{krb5_free_safe}{void}{\funcinout}
\funcarg{krb5_context}{context}
\funcarg{krb5_safe *}{val}
\end{funcdecl}

Frees the safe application data \funcparam{val} that is allocated with
\funcparam{decode_krb5_safe}. 


\begin{funcdecl}{krb5_free_priv}{void}{\funcinout}
\funcarg{krb5_context}{context}
\funcarg{krb5_priv *}{val}
\end{funcdecl}

Frees the private data  \funcparam{val} that has been allocated from
\funcname{decode_krb5_priv}. 

\begin{funcdecl}{krb5_free_priv_enc_part}{void}{\funcinout}
\funcarg{krb5_context}{context}
\funcarg{krb5_priv_enc_part *}{val}
\end{funcdecl}

Frees the private encoded part \funcparam{val} that has been allocated from
\funcname{decode_krb5_enc_priv_part}. 

\begin{funcdecl}{krb5_free_cred}{void}{\funcinout}
\funcarg{krb5_context}{context}
\funcarg{krb5_cred *}{val}
\end{funcdecl}

Frees the credential \funcparam{val}.

\begin{funcdecl}{krb5_free_creds}{void}{\funcinout}
\funcarg{krb5_context}{context}
\funcarg{krb5_creds *}{val}
\end{funcdecl}

Calls \funcname{krb5_free_cred_contents} with \funcparam{val} as the
argument. \funcparam{val} is freed as well.

\begin{funcdecl}{krb5_free_cred_contents}{void}{\funcinout}
\funcarg{krb5_context}{context}
\funcarg{krb5_creds *}{val}
\end{funcdecl}

The function zeros out the session key stored in the credential and then
frees the credentials structures. The argument \funcparam{val} is
{\bf not} freed.


\begin{funcdecl}{krb5_free_cred_enc_part}{void}{\funcinout}
\funcarg{krb5_context}{context}
\funcarg{krb5_cred_enc_part *}{val}
\end{funcdecl}

Frees the addresses and ticket_info elements of
\funcparam{val}. \funcparam{val} is {\bf not} freed by this routine.

\begin{funcdecl}{krb5_free_checksum}{void}{\funcinout}
\funcarg{krb5_context}{context}
\funcarg{krb5_checksum *}{val}
\end{funcdecl}

The checksum and the pointer \funcparam{val} are both freed. 

\begin{funcdecl}{krb5_free_keyblock}{void}{\funcinout}
\funcarg{krb5_context}{context}
\funcarg{krb5_keyblock *}{val}
\end{funcdecl}

The keyblock contents of \funcparam{val} are zeroed and the memory
freed. The pointer \funcparam{val} is freed as well.

\begin{funcdecl}{krb5_free_pa_data}{void}{\funcinout}
\funcarg{krb5_context}{context}
\funcarg{krb5_pa_data **}{val}
\end{funcdecl}

Frees the contents of \funcparam{*val}. \funcparam{val} is freed as
well.

\begin{funcdecl}{krb5_free_ap_rep_enc_part}{void}{\funcinout}
\funcarg{krb5_context}{context}
\funcarg{krb5_ap_rep_enc_part *}{val}
\end{funcdecl}

Frees the subkey keyblock (if set) as well as \funcparam{val} that has
been allocated from \funcname{krb5_rd_rep} or \funcname{krb5_send_auth}.

\begin{funcdecl}{krb5_free_tkt_authent}{void}{\funcinout}
\funcarg{krb5_context}{context}
\funcarg{krb5_tkt_authent *}{val}
\end{funcdecl}

Frees the ticket and authenticator portions of \funcparam{val}. The
pointer \funcparam{val} is freed as well.

\begin{funcdecl}{krb5_free_pwd_data}{void}{\funcinout}
\funcarg{krb5_context}{context}
\funcarg{passwd_pwd_data *}{val}
\end{funcdecl}

Frees the pwd_data \funcparam{val} that has been allocated from
\funcname{decode_krb5_pwd_data}. 

\begin{funcdecl}{krb5_free_pwd_sequences}{void}{\funcinout}
\funcarg{krb5_context}{context}
\funcarg{passwd_phrase_element **}{val}
\end{funcdecl}

Frees the passwd_phrase_element \funcparam{val}. This is usually called
from \funcname{krb5_free_pwd_data}.

\begin{funcdecl}{krb5_free_realm_tree}{void}{\funcinout}
\funcarg{krb5_context}{context}
\funcarg{krb5_principal *}{realms}
\end{funcdecl}

Frees the realms tree \funcparam{realms} returned by
\funcname{krb5_walk_realm_tree}.

\begin{funcdecl}{krb5_free_tgt_creds}{void}{\funcinout}
\funcarg{krb5_context}{context}
\funcarg{krb5_creds **}{tgts}
\end{funcdecl}

Frees the TGT credentials \funcparam{tgts} returned by
\funcname{krb5_get_cred_from_kdc}.



\subsection{Operating-system specific functions}
The operating-system specific functions provide an interface between the
other parts of the \libname{libkrb5.a} libraries and the operating system.

Beware! Any of these are allowed to be implemented as macros.

The following global symbols are provided in \libname{libos.a}.  If you
wish to substitute for any of them, you must substitute for all of them
(they are all declared and initialized in the same object file):
\begin{itemize}
% These come from src/lib/osconfig.c
\item extern char *\globalname{krb5_config_file}: name of configuration file
\item extern char *\globalname{krb5_trans_file}: name of hostname/realm
name translation file
\item extern char *\globalname{krb5_defkeyname}: default name of key
table file
\item extern char *\globalname{krb5_lname_file}: name of aname/lname
translation database
\item extern int \globalname{krb5_max_dgram_size}: maximum allowable
datagram size
\item extern int \globalname{krb5_max_skdc_timeout}: maximum
per-message KDC reply timeout
\item extern int \globalname{krb5_skdc_timeout_shift}: shift factor
(bits) to exponentially back-off the KDC timeouts
\item extern int \globalname{krb5_skdc_timeout_1}: initial KDC timeout
\item extern char *\globalname{krb5_kdc_udp_portname}: name of KDC UDP port
\item extern char *\globalname{krb5_default_pwd_prompt1}: first prompt
for password reading.
\item extern char *\globalname{krb5_default_pwd_prompt2}: second prompt

\end{itemize}

\begin{funcdecl}{krb5_read_password}{krb5_error_code}{\funcin}
\funcarg{char *}{prompt}
\funcarg{char *}{prompt2}
\funcout
\funcarg{char *}{return_pwd}
\funcinout
\funcarg{int *}{size_return}
\end{funcdecl}

Read a password from the keyboard.  The first \funcparam{*size_return}
bytes of the password entered are returned in \funcparam{return_pwd}.
If fewer than \funcparam{*size_return} bytes are typed as a password,
the remainder of \funcparam{return_pwd} is zeroed.  Upon success, the
total number of bytes filled in is stored in \funcparam{*size_return}.

\funcparam{prompt} is used as the prompt for the first reading of a password.
It is printed to the terminal, and then a password is read from the
keyboard.  No newline or spaces are emitted between the prompt and the
cursor, unless the newline/space is included in the prompt.

If \funcparam{prompt2} is a null pointer, then the password is read
once.  If \funcparam{prompt2} is set, then it is used as a prompt to
read another password in the same manner as described for
\funcparam{prompt}.  After the second password is read, the two
passwords are compared, and an error is returned if they are not
identical.

Echoing is turned off when the password is read.

If there is an error in reading or verifying the password, an error code
is returned; else zero is returned.

\begin{funcdecl}{krb5_lock_file}{krb5_error_code}{\funcvoid}
\funcarg{FILE *}{filep}
\funcarg{char *}{pathname}
\funcarg{int}{mode}
\end{funcdecl}

Attempts to lock the file in the given \funcparam{mode}; returns 0 for a
successful lock, or an error code otherwise.

The caller should arrange that both \funcparam{filep} and
\funcparam{pathname} refer to the same
file.  The implementation may use whichever is more convenient.

Modes are given in {\tt <krb5/libos.h>}


\begin{funcdecl}{krb5_unlock_file}{krb5_error_code}{\funcvoid}
\funcarg{FILE *}{filep}
\funcarg{char *}{pathname}
\end{funcdecl}

Attempts to (completely) unlock the file.  Returns 0 if successful,
or an error code otherwise.

The caller should arrange that both \funcparam{filep} and
\funcparam{pathname} refer to the same file.  The implementation may
use whichever is more convenient.

\begin{funcdecl}{krb5_timeofday}{krb5_error_code}{\funcout}
\funcarg{krb5_int32 *}{timeret}
\end{funcdecl}

Retrieves the system time of day, in seconds since the local system's
epoch.
[The ASN.1 encoding routines must convert this to the standard ASN.1
encoding as needed]

\begin{funcdecl}{krb5_ms_timeofday}{krb5_error_code}{\funcout}
\funcarg{int32 *}{seconds}
\funcarg{int16 *}{milliseconds}
\end{funcdecl}

Retrieves the system time of day, in seconds since the local system's
epoch.
[The ASN.1 encoding routines must convert this to the standard ASN.1
encoding as needed]

The seconds portion is returned in \funcparam{*seconds}, the
milliseconds portion in \funcparam{*milliseconds}.

\begin{funcdecl}{krb5_net_read}{int}{\funcin}
\funcarg{int}{fd}
\funcout
\funcarg{char *}{buf}
\funcin
\funcarg{int}{len}
\end{funcdecl}

Like read(2), but guarantees that it reads as much as was requested
or returns -1 and sets errno.

(make sure your sender will send all the stuff you are looking for!)
Only useful on stream sockets and pipes.

\begin{funcdecl}{krb5_net_write}{int}{\funcin}
\funcarg{int}{fd}
\funcarg{const char *}{buf}
\funcarg{int}{len}
\end{funcdecl}

Like write(2), but guarantees that it writes as much as was requested
or returns -1 and sets errno.

(make sure your sender will send all the stuff you are looking for!)
Only useful on stream sockets and pipes.

\begin{funcdecl}{krb5_os_localaddr}{krb5_error_code}{\funcout}
\funcarg{krb5_address ***}{addr}
\end{funcdecl}

Return all the protocol addresses of this host.

Compile-time configuration flags will indicate which protocol family
addresses might be returned.
\funcparam{*addr} is filled in to point to an array of address pointers,
terminated by a null pointer.  All the storage pointed to is allocated
and should be freed by the caller with \funcname{krb5_free_address}
when no longer needed.


\begin{funcdecl}{krb5_sendto_kdc}{krb5_error_code}{\funcin}
\funcarg{krb5_data *}{send}
\funcarg{krb5_data *}{realm}
\funcout
\funcarg{krb5_data *}{receive}
\end{funcdecl}

Send the message \funcparam{send} to a KDC for realm \funcparam{realm} and
return the response (if any) in \funcparam{receive}.

If the message is sent and a response is received, 0 is returned,
otherwise an error code is returned.

The storage for \funcparam{receive} is allocated and should be freed by
the caller when finished.

\begin{funcdecl}{krb5_get_krbhst}{krb5_error_code}{\funcin}
\funcarg{krb5_data *}{realm}
\funcout
\funcarg{char ***}{hostlist}
\end{funcdecl}

Figures out the Kerberos server names for the given \funcparam{realm},
filling in
\funcparam{hostlist} with a
pointer to an argv[] style list of names, terminated with a null
pointer.
 
If \funcparam{realm} is unknown, the filled-in pointer is set to NULL.

The pointer array and strings pointed to are all in allocated storage,
and should be freed by the caller when finished.

Returns system errors.

\begin{funcdecl}{krb5_free_krbhst}{krb5_error_code}{\funcin}
\funcarg{char **}{hostlist}
\end{funcdecl}

Frees the storage taken by a host list returned by \funcname{krb5_get_krbhst}.

\begin{funcdecl}{krb5_aname_to_localname}{krb5_error_code}{\funcin}
\funcarg{krb5_principal}{aname}
\funcarg{int}{lnsize}
\funcout
\funcarg{char *}{lname}
\end{funcdecl}

Converts a principal name \funcparam{aname} to a local name suitable for use by
programs wishing a translation to an environment-specific name (e.g.
user account name).

\funcparam{lnsize} specifies the maximum length name that is to be filled into
\funcparam{lname}.
The translation will be null terminated in all non-error returns.

Returns system errors.

\begin{funcdecl}{krb5_get_default_realm}{krb5_error_code}{\funcin}
\funcarg{int}{lnsize}
\funcout
\funcarg{char *}{lrealm}
\end{funcdecl}

Retrieves the default realm to be used if no user-specified realm is
available (e.g. to interpret a user-typed principal name with the
realm omitted for convenience).

\funcparam{lnsize} specifies the maximum length name that is to be filled into
\funcparam{lrealm}.

Returns system errors.

\begin{funcdecl}{krb5_get_host_realm}{krb5_error_code}{\funcin}
\funcarg{char *}{host}
\funcout
\funcarg{char ***}{realmlist}
\end{funcdecl}

Figures out the Kerberos realm names for \funcparam{host}, filling in
\funcparam{realmlist} with a
pointer to an argv[] style list of names, terminated with a null pointer.
 
If \funcparam{host} is NULL, the local host's realms are determined.

If there are no known realms for the host, the filled-in pointer is set
to NULL.

The pointer array and strings pointed to are all in allocated storage,
and should be freed by the caller when finished.

Returns system errors.

\begin{funcdecl}{krb5_free_host_realm}{krb5_error_code}{\funcin}
\funcarg{char **}{realmlist}
\end{funcdecl}

Frees the storage taken by a \funcparam{realmlist} returned by
\funcname{krb5_get_local_realm}.

\begin{funcdecl}{krb5_kuserok}{krb5_boolean}{\funcin}
\funcarg{krb5_principal}{principal}
\funcarg{const char *}{luser}
\end{funcdecl}

Given a Kerberos principal \funcparam{principal}, and a local username
\funcparam{luser},
determine whether user is authorized to login to the account \funcparam{luser}.
Returns TRUE if authorized, FALSE if not authorized.

\begin{funcdecl}{krb5_random_confounder}{krb5_confounder}{\funcvoid}
\end{funcdecl}

Generate a random confounder.


\appendix
\cleardoublepage
\printindex
\end{document}
