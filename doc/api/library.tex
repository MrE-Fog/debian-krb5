\documentstyle[newcen,fixunder,functions,changebar,twoside,fancyheadings,backslash]{article}
\setlength{\oddsidemargin}{0.25in}
\setlength{\evensidemargin}{-0.25in}
\setlength{\topmargin}{-.5in}
\setlength{\textheight}{9in}
\setlength{\parskip}{.1in}
\setlength{\parindent}{2em}
\setlength{\textwidth}{6.25in}
\makeindex
\newif\ifdraft
\drafttrue
\typein{Draft flag? (type \backslash draftfalse<CR> if not draft...)}
\ifdraft
\pagestyle{fancy}
\lhead[\fancyplain{}\thepage]{\fancyplain{}{\sl \leftmark}}
\rhead[\fancyplain{}{\sl \leftmark}]{\fancyplain{}\thepage}
\cfoot{{\bf DRAFT---DO NOT REDISTRIBUTE}}
\else\pagestyle{headings}\fi
\begin{document}
\thispagestyle{empty}
\begin{center}
{\Huge Kerberos V5 application programming library}
\ifdraft \\ {\Large DRAFT---\today}\fi
\end{center}
\section{libkrb5.a functions}
This section describes the functions provided in the \libname{libkrb5.a}
library.  The library is built from several pieces, mostly for convenience in
programming, maintenance, and porting.

\ifdraft\sloppy\fi

\subsection{Main functions}
The main functions deal with the nitty-gritty details: verifying
tickets, creating authenticators, and the like.

\begin{funcdecl}[krb5_encode_kdc_rep]{krb5_error_code}{\funcin}
\funcarg{krb5_msgtype}{type}
\funcarg{krb5_enc_kdc_rep_part *}{encpart}
\funcarg{krb5_keyblock *}{client_key}
\funcinout
\funcarg{krb5_kdc_rep *}{dec_rep}
\funcout
\funcarg{krb5_data *}{enc_rep}
\end{funcdecl}

Takes KDC rep parts in \funcparam{*rep} and \funcparam{*encpart}, and
formats it into \funcparam{*enc_rep}, using message type \funcparam{type}
and encryption key \funcparam{client_key} and encryption type
\funcparam{dec_rep{\ptsto}etype}.

\funcparam{enc_rep{\ptsto}data} will point to  allocated storage upon
non-error return; the caller should free it when finished.

Returns system errors.

\begin{funcdecl}[krb5_decode_kdc_rep]{krb5_error_code}{\funcin}
\funcarg{krb5_data *}{enc_rep}
\funcarg{krb5_keyblock *}{key}
\funcarg{krb5_enctype}{etype}
\funcout
\funcarg{krb5_kdc_rep **}{dec_rep}
\end{funcdecl}

Takes a KDC_REP message and decrypts encrypted part using
\funcparam{etype} and \funcparam{*key}, putting result in \funcparam{*rep}.
The pointers in \funcparam{dec_rep}
are all set to allocated storage which should be freed by the caller
when finished with the response (by using \funcname{krb5_free_kdc_rep}).


If the response isn't a KDC_REP (tgs or as), it returns an error from
the decoding routines (usually ISODE_50_LOCAL_ERR_BADDECODE).

Returns errors from encryption routines, system errors.



\subsection{Credentials cache functions}
The credentials cache functions (some of which are macros which call to
specific types of credentials caches) deal with storing credentials
(tickets, session keys, and other identifying information) in a
semi-permanent store for later use by different programs.

\subsubsection{Per-type functions}
The following entry points must be implemented for each type of
credentials cache; however, applications are not expected to have a need
to call either \funcname{krb5_cc_resolve_internal} or
\funcname{krb5_cc_gennew_internal}.


\begin{funcdecl}{krb5_error_code}{krb5_cc_resolve_internal}{\funcout}
\funcarg{krb5_ccache *}{id}
\funcin
\funcarg{char *}{residual}
\end{funcdecl}

Creates a credentials cache named by \funcparam{residual} (which may be
interpreted differently by each type of ccache).  The cache is not
opened, but the cache name is held in reserve.

\begin{funcdecl}{krb5_error_code}{krb5_cc_gennew_internal}{\funcout}
\funcarg{krb5_ccache *}{id}
\end{funcdecl}

Creates a new credentials cache whose name is guaranteed to be
unique.  The cache is not opened. \funcparam{*id} is
filled in with a \datatype{krb5_ccache} which may be used in subsequent
calls to ccache functions.

\begin{funcdecl}{krb5_cc_initialize}{krb5_error_code}{\funcinout}
\funcarg{krb5_ccache}{id}
\funcin
\funcarg{krb5_principal}{primary_principal}
\end{funcdecl}

Creates/refreshes a credentials cache identified by \funcparam{id} with
primary principal set to \funcparam{primary_principal}.
If the credentials cache already exists, its contents are destroyed.

Errors:  permission errors, system errors.

Modifies: cache identified by \funcparam{id}.

\begin{funcdecl}{krb5_cc_destroy}{krb5_error_code}{\funcin}
\funcarg{krb5_ccache}{id}
\end{funcdecl}

Destroys the credentials cache identified by \funcparam{id}.
Requires that the credentials cache exist.

Errors:  permission errors.

\begin{funcdecl}{krb5_cc_close}{krb5_error_code}{\funcinout}
\funcarg{krb5_ccache}{id}
\end{funcdecl}

Closes the credentials cache \funcparam{id}, invalidates \funcparam{id},
and releases any other resources acquired during use of the credentials cache.
Requires that \funcparam{id} identify a valid credentials cache.


\begin{funcdecl}{krb5_cc_store_cred}{krb5_error_code}{\funcin}
\funcarg{krb5_ccache}{id}
\funcarg{krb5_credentials *}{creds}
\end{funcdecl}

Stores \funcparam{creds} in the cache \funcparam{id}, tagged with
\funcparam{creds{\ptsto}client}.
Requires that \funcparam{id} identify a valid credentials cache.

Errors: permission errors, storage failure errors.

\begin{funcdecl}{krb5_cc_retrieve_cred}{krb5_error_code}{\funcin}
\funcarg{krb5_ccache}{id}
\funcarg{krb5_flags}{whichfields}
\funcarg{krb5_credentials *}{mcreds}
\funcout
\funcarg{krb5_credentials *}{creds}
\end{funcdecl}

Searches the cache \funcparam{id} for credentials matching
\funcparam{mcreds}.  The fields which are to be matched are specified by
set bits in \funcparam{whichfields}, and always include the principal
name \funcparam{mcreds{\ptsto}server}.
Requires that \funcparam{id} identify a valid credentials cache.

If at least one match is found, one of the matching credentials is
returned in \funcparam{*creds}. XXX free the return creds?

Errors: error code if no matches found.

\begin{funcdecl}{krb5_cc_get_principal}{krb5_error_code}{\funcin}
\funcarg{krb5_ccache}{id}
\funcarg{krb5_principal *}{principal}
\end{funcdecl}

Retrieves the primary principal of the credentials cache (as
set by the \funcname{krb5_cc_initialize} request)
The primary principal is filled into \funcparam{*principal}; the caller
should release this memory by calling \funcname{krb5_free_principal} on
\funcparam{*principal} when finished.

Requires that \funcparam{id} identify a valid credentials cache.

\begin{funcdecl}{krb5_cc_start_seq_get}{krb5_error_code}{\funcin}
\funcarg{krb5_ccache}{id}
\funcout
\funcarg{krb5_cc_cursor *}{cursor}
\end{funcdecl}

Prepares to sequentially read every set of cached credentials.
Requires that \funcparam{id} identify a valid credentials cache opened by
\funcname{krb5_cc_open}.
\funcparam{cursor} is filled in with a cursor to be used in calls to
\funcname{krb5_cc_next_cred}.

\begin{funcdecl}{krb5_cc_next_cred}{krb5_error_code}{\funcin}
\funcarg{krb5_ccache}{id}
\funcout
\funcarg{krb5_credentials *}{creds}
\funcinout
\funcarg{krb5_cc_cursor *}{cursor}
\end{funcdecl}

Fetches the next entry from \funcparam{id}, returning its values in
\funcparam{*creds}, and updates \funcparam{*cursor} for the next request.
Requires that \funcparam{id} identify a valid credentials cache and
\funcparam{*cursor} be a cursor returned by
\funcname{krb5_cc_start_seq_get} or a subsequent call to
\funcname{krb5_cc_next_cred}.

Errors: error code if no more cache entries.

\begin{funcdecl}{krb5_cc_end_seq_get}{krb5_error_code}{\funcin}
\funcarg{krb5_ccache}{id}
\funcarg{krb5_cc_cursor *}{cursor}
\end{funcdecl}

Finishes sequential processing mode and invalidates \funcparam{*cursor}.
\funcparam{*cursor} must never be re-used after this call.

Requires that \funcparam{id} identify a valid credentials cache and
\funcparam{*cursor} be a cursor returned by
\funcname{krb5_cc_start_seq_get} or a subsequent call to
\funcname{krb5_cc_next_cred}.

Errors: may return error code if \funcparam{*cursor} is invalid.


\begin{funcdecl}{krb5_cc_remove_cred}{krb5_error_code}{\funcin}
\funcarg{krb5_ccache}{id}
\funcarg{krb5_flags}{which}
\funcarg{krb5_credentials *}{cred}
\end{funcdecl}

Removes any credentials from \funcparam{id} which match the principal
name {cred{\ptsto}server} and the fields in \funcparam{cred} masked by
\funcparam{which}.
Requires that \funcparam{id} identify a valid credentials cache.

Errors: returns error code if nothing matches; returns error code if
couldn't delete.

\begin{funcdecl}{krb5_cc_set_flags}{krb5_error_code}{\funcin}
\funcarg{krb5_ccache}{id}
\funcarg{krb5_flags}{flags}
\end{funcdecl}

Sets the flags on the cache \funcparam{id} to \funcparam{flags}.


\subsubsection{Glue functions}
The following functions are implemented in the base library and serve to
glue together the various types of credentials caches.


\begin{funcdecl}{krb5_cc_resolve}{krb5_error_code}{\funcin}
\funcarg{char *}{string_name}
\funcout
\funcarg{krb5_ccache *}{id}
\end{funcdecl}

Fills in \funcparam{id} with a ccache identifier which corresponds to
the name in \funcparam{string_name}.  The cache is left unopened.

Requires that \funcparam{string_name} be of the form ``type:residual'' and
``type'' is a type known to the library.

\begin{funcdecl}{krb5_cc_generate_new}{krb5_error_code}{\funcin}
\funcarg{krb5_cc_ops *}{ops}
\funcout
\funcarg{krb5_ccache *}{id}
\end{funcdecl}


Fills in \funcparam{id} with a unique ccache identifier of a type defined by
\funcparam{ops}.  The cache is left unopened.

\begin{funcdecl}{krb5_cc_register}{krb5_error_code}{\funcin}
\funcarg{krb5_cc_ops *}{ops}
\funcarg{krb5_boolean}{override}
\end{funcdecl}

Adds a new cache type identified and implemented by \funcparam{ops} to
the set recognized by \funcname{krb5_cc_resolve}.
If \funcparam{override} is FALSE, a ticket cache type named
\funcparam{ops{\ptsto}prefix} must not be known.

\begin{funcdecl}{krb5_cc_get_name}{char *}{\funcin}
\funcarg{krb5_ccache}{id}
\end{funcdecl}

Returns the name of the ccache denoted by \funcparam{id}.

\begin{funcdecl}{krb5_cc_default_name}{char *}{\funcvoid}
\end{funcdecl}

Returns the name of the default credentials cache; this may be equivalent to
{\funcfont getenv}({\tt "KRB5CCACHE"}) with an appropriate fallback.

\begin{funcdecl}{krb5_cc_default }{krb5_error_code}{\funcout}
\funcarg{krb5_ccache *}{ccache}
\end{funcdecl}

Equivalent to {\funcfont krb5_cc_resolve}({\funcfont
krb5_cc_default_name}(), \funcparam{ccache}).



\subsection{Replay cache functions}
The replay cache functions deal with verifying that AP_REQ's do not
contain duplicate authenticators; the storage must be non-volatile for
the site-determined validity period of authenticators.

Each replay cache has a string ``name'' associated with it.  The use of
this name is dependent on the underlying caching strategy (for
file-based things, it would be a cache file name).  The
caching strategy should use non-volatile storage so that replay
integrity can be maintained across system failures.

\subsubsection{Per-type functions}
The following entry points must be implemented for each type of
credentials cache.

\begin{funcdecl}{krb5_rc_initialize}{krb5_error_code}{\funcin}
\funcarg{krb5_rcache}{id}
\funcarg{krb5_deltat}{auth_lifespan}
\end{funcdecl}

Creates/refreshes the replay cache identified by \funcparam{id} and sets its
authenticator lifespan to \funcparam{auth_lifespan}.  If the 
replay cache already exists, its contents are destroyed.

Errors: permission errors, system errors

\begin{funcdecl}{krb5_rc_recover}{krb5_error_code}{\funcin}
\funcarg{krb5_rcache}{id}
\end{funcdecl}
Attempts to recover the replay cache \funcparam{id}, (presumably after a
system crash or server restart).

Errors: error indicating that no cache was found to recover

\begin{funcdecl}{krb5_rc_destroy}{krb5_error_code}{\funcin}
\funcarg{krb5_rcache}{id}
\end{funcdecl}

Destroys the replay cache \funcparam{id}.
Requires that \funcparam{id} identifies a valid replay cache.

Errors: permission errors.

\begin{funcdecl}{krb5_rc_close}{krb5_error_code}{\funcin}
\funcarg{krb5_rcache}{id}
\end{funcdecl}

Closes the replay cache \funcparam{id}, invalidates \funcparam{id},
and releases any other resources acquired during use of the replay cache.
Requires that \funcparam{id} identifies a valid replay cache.

Errors: permission errors

\begin{funcdecl}{krb5_rc_store}{krb5_error_code}{\funcin}
\funcarg{krb5_rcache}{id}
\funcarg{krb5_dont_replay *}{rep}
\end{funcdecl}
Stores \funcparam{rep} in the replay cache \funcparam{id}.
Requires that \funcparam{id} identifies a valid replay cache.

Returns KRB5KRB_AP_ERR_REPEAT if \funcparam{rep} is already in the
cache.  May also return permission errors, storage failure errors.

\begin{funcdecl}{krb5_rc_expunge}{krb5_error_code}{\funcin}
\funcarg{krb5_rcache}{id}
\end{funcdecl}
Removes all expired replay information (i.e. those entries which are
older than then authenticator lifespan of the cache) from the cache
\funcparam{id}.  Requires that \funcparam{id} identifies a valid replay
cache.

Errors: permission errors.

\begin{funcdecl}{krb5_rc_get_lifespan}{krb5_error_code}{\funcin}
\funcarg{krb5_rcache}{id}
\funcout
\funcarg{krb5_deltat *}{auth_lifespan}
\end{funcdecl}
Fills in \funcparam{auth_lifespan} with the lifespan of
the cache \funcparam{id}.
Requires that \funcparam{id} identifies a valid replay cache.

\begin{funcdecl}{krb5_rc_resolve}{krb5_error_code}{\funcinout}
\funcarg{krb5_rcache}{id}
\funcin
\funcarg{char *}{name}
\end{funcdecl}

Initializes private data attached to \funcparam{id}.  This function MUST
be called before the other per-replay cache functions.

Requires that \funcparam{id} points to allocated space, with an
initialized \funcparam{id{\ptsto}ops} field.

Returns:  allocation errors.


\begin{funcdecl}{krb5_rc_get_name}{char *}{\funcin}
\funcarg{krb5_rcache}{id}
\end{funcdecl}

Returns the name (excluding the type) of the rcache \funcparam{id}.
Requires that \funcparam{id} identifies a valid replay cache.

\subsubsection{Glue functions}
The following functions are implemented in the base library and serve to
glue together the various types of replay caches.

\begin{funcdecl}{krb5_rc_resolve_full}{krb5_error_code}{\funcinout}
\funcarg{krb5_rcache *}{id}
\funcin
\funcarg{char *}{string_name}
\end{funcdecl}

\funcparam{id} is filled in to identify a replay cache which
corresponds to the name in \funcparam{string_name}.  The cache is not opened.
Requires that \funcparam{string_name} be of the form ``type:residual''
and that ``type'' is a type known to the library.

Errors: error if cannot resolve name.

\begin{funcdecl}{krb5_rc_register_type}{krb5_error_code}{\funcin}
\funcarg{krb5_rc_ops *}{ops}
\end{funcdecl}
Adds a new replay cache type implemented and identified by
\funcparam{ops} to the set recognized by
\funcname{krb5_rc_resolve}.  Requires that a ticket cache type named
\funcparam{ops{\ptsto}prefix} is not yet known.


\begin{funcdecl}{krb5_rc_default_name}{char *}{\funcvoid}
\end{funcdecl}
Returns  the name of the default replay cache; this may be equivalent to
\funcnamenoparens{getenv}({\tt "KRB5RCACHE"}) with an appropriate fallback.

\begin{funcdecl}{krb5_rc_default_type}{char *}{\funcvoid}
\end{funcdecl}

Returns the type of the default replay cache.

\begin{funcdecl}{krb5_rc_default}{krb5_error_code}{\funcinout}
\funcarg{krb5_rcache *}{id}
\end{funcdecl}
Equivalent to \funcnamenoparens{krb5_rc_resolve_full}(\funcparam{id},
\funcnamenoparens{strcat}(\funcnamenoparens{strcat}(\funcname{krb5_rc_default_type},``:''),
\funcname{krb5_rc_default_name})) (except of course you can't do the
strcat's with the return values\ldots).


\subsection{Key table functions}
The key table functions deal with storing and retrieving service keys
for use by unattended services which participate in authentication exchanges.

Keytab routines are all be atomic.  Every routine that acquires
a non-sharable resource releases it before it returns. 

All keytab types support multiple concurrent sequential scans.

The order of values returned from \funcname{krb5_kt_next_entry} is
unspecified.

Although the ``right thing'' should happen if the program aborts
abnormally, a close routine, \funcname{krb5_kt_free_entry},  is provided
for freeing resources, etc.  People should use the close routine when
they are finished.

\begin{funcdecl}{krb5_kt_register}{krb5_error_code}{\funcinout}
\funcarg{krb5_context}{context}
\funcin
\funcarg{krb5_kt_ops *}{ops}
\end{funcdecl}


Adds a new ticket cache type to the set recognized by
\funcname{krb5_kt_resolve}.
Requires that a keytab type named \funcparam{ops{\ptsto}prefix} is not
yet known.

An error is returned if \funcparam{ops{\ptsto}prefix} is already known.

\begin{funcdecl}{krb5_kt_resolve}{krb5_error_code}{\funcinout}
\funcarg{krb5_context}{context}
\funcin
\funcarg{const char *}{string_name}
\funcout
\funcarg{krb5_keytab *}{id}
\end{funcdecl}

Fills in \funcparam{*id} with a handle identifying the keytab with name
``string_name''.  The keytab is not opened.
Requires that \funcparam{string_name} be of the form ``type:residual'' and
``type'' is a type known to the library.

Errors: badly formatted name.
		
\begin{funcdecl}{krb5_kt_default_name}{krb5_error_code}{\funcinout}
\funcarg{krb5_context}{context}
\funcin
\funcarg{char *}{name}
\funcarg{int}{namesize}
\end{funcdecl}

\funcparam{name} is filled in with the first \funcparam{namesize} bytes of
the name of the default keytab.
If the name is shorter than \funcparam{namesize}, then the remainder of
\funcparam{name} will be zeroed.


\begin{funcdecl}{krb5_kt_default}{krb5_error_code}{\funcinout}
\funcarg{krb5_context}{context}
\funcin
\funcarg{krb5_keytab *}{id}
\end{funcdecl}

Fills in \funcparam{id} with a handle identifying the default keytab.

\begin{funcdecl}{krb5_kt_read_service_key}{krb5_error_code}{\funcinout}
\funcarg{krb5_context}{context}
\funcin
\funcarg{krb5_pointer}{keyprocarg}
\funcarg{krb5_principal}{principal}
\funcarg{krb5_kvno}{vno}
\funcarg{krb5_keytype}{keytype}
\funcout
\funcarg{krb5_keyblock **}{key}
\end{funcdecl}

If \funcname{keyprocarg} is not NULL, it is taken to be a
\datatype{char *} denoting the name of a keytab.  Otherwise, the default
keytab will be used.
The keytab is opened and searched for the entry identified by
\funcparam{principal}, \funcparam{keytype}, and \funcparam{vno}, 
returning the resulting key
in \funcparam{*key} or returning an error code if it is not found. 

\funcname{krb5_free_keyblock} should be called on \funcparam{*key} when
the caller is finished with the key.

Returns an error code if the entry is not found.

\begin{funcdecl}{krb5_kt_add_entry}{krb5_error_code}{\funcinout}
\funcarg{krb5_context}{context}
\funcin
\funcarg{krb5_keytab}{id}
\funcarg{krb5_keytab_entry *}{entry}
\end{funcdecl}

Calls the keytab-specific add routine \funcname{krb5_kt_add_internal}
with the same function arguments.  If this routine is not available,
then KRB5_KT_NOWRITE is returned.

\begin{funcdecl}{krb5_kt_remove_entry}{krb5_error_code}{\funcinout}
\funcarg{krb5_context}{context}
\funcin
\funcarg{krb5_keytab}{id}
\funcarg{krb5_keytab_entry *}{entry}
\end{funcdecl}

Calls the keytab-specific remove routine
\funcname{krb5_kt_remove_internal} with the same function arguments.
If this routine is not available, then KRB5_KT_NOWRITE is returned.

\begin{funcdecl}{krb5_kt_get_name}{krb5_error_code}{\funcinout}
\funcarg{krb5_context}{context}
\funcarg{krb5_keytab}{id}
\funcout
\funcarg{char *}{name}
\funcin
\funcarg{int}{namesize}
\end{funcdecl}

\funcarg{name} is filled in with the first \funcparam{namesize} bytes of
the name of the keytab identified by \funcname{id}.
If the name is shorter than \funcparam{namesize}, then \funcarg{name}
will be null-terminated.

\begin{funcdecl}{krb5_kt_close}{krb5_error_code}{\funcinout}
\funcarg{krb5_context}{context}
\funcarg{krb5_keytab}{id}
\end{funcdecl}

Closes the keytab identified by \funcparam{id} and invalidates
\funcparam{id}, and releases any other resources acquired during use of
the key table.

Requires that \funcparam{id} identifies a keytab.

\begin{funcdecl}{krb5_kt_get_entry}{krb5_error_code}{\funcinout}
\funcarg{krb5_context}{context}
\funcarg{krb5_keytab}{id}
\funcin
\funcarg{krb5_principal}{principal}
\funcarg{krb5_kvno}{vno}
\funcarg{krb5_keytype}{keytype}
\funcout
\funcarg{krb5_keytab_entry *}{entry}
\end{funcdecl}

\begin{sloppypar}
Searches the keytab identified by \funcparam{id} for an entry whose
principal matches \funcparam{principal}, whose keytype matches 
\funcparam{keytype}, and
whose key version number matches \funcparam{vno}.  If \funcparam{vno} is
zero, the first entry whose principal matches is returned.
\end{sloppypar}

Returns an error code if no suitable entry is found.  If an entry is
found, the entry is returned in \funcparam{*entry}; its contents should
be deallocated by calling \funcname{krb5_kt_free_entry} when no longer
needed.

\begin{funcdecl}{krb5_kt_free_entry}{krb5_error_code}{\funcinout}
\funcarg{krb5_context}{context}
\funcarg{krb5_keytab_entry *}{entry}
\end{funcdecl}

Releases all storage allocated for \funcparam{entry}, which must point
to a structure previously filled in by \funcname{krb5_kt_get_entry} or
\funcname{krb5_kt_next_entry}.

\begin{funcdecl}{krb5_kt_start_seq_get}{krb5_error_code}{\funcinout}
\funcarg{krb5_context}{context}
\funcarg{krb5_keytab}{id}
\funcout
\funcarg{krb5_kt_cursor *}{cursor}
\end{funcdecl}

Prepares to read sequentially every key in the keytab identified by
\funcparam{id}.
\funcparam{cursor} is filled in with a cursor to be used in calls to
\funcname{krb5_kt_next_entry}.

\begin{funcdecl}{krb5_kt_next_entry}{krb5_error_code}{\funcinout}
\funcarg{krb5_context}{context}
\funcarg{krb5_keytab}{id}
\funcout
\funcarg{krb5_keytab_entry *}{entry}
\funcinout
\funcarg{krb5_kt_cursor}{cursor}
\end{funcdecl}

Fetches the ``next'' entry in the keytab, returning it in
\funcparam{*entry}, and updates \funcparam{*cursor} for the next
request.  If the keytab changes during the sequential get, an error is
guaranteed.  \funcparam{*entry} should be freed after use by calling
\funcname{krb5_kt_free_entry}.

Requires that \funcparam{id} identifies a valid keytab.  and
\funcparam{*cursor} be a cursor returned by
\funcname{krb5_kt_start_seq_get} or a subsequent call to
\funcname{krb5_kt_next_entry}.

Errors: error code if no more cache entries or if the keytab changes.

\begin{funcdecl}{krb5_kt_end_seq_get}{krb5_error_code}{\funcinout}
\funcarg{krb5_context}{context}
\funcarg{krb5_keytab}{id}
\funcarg{krb5_kt_cursor *}{cursor}
\end{funcdecl}

Finishes sequential processing mode and invalidates \funcparam{cursor},
which must never be re-used after this call.

Requires that \funcparam{id} identifies a valid keytab  and
\funcparam{*cursor} be a cursor returned by
\funcname{krb5_kt_start_seq_get} or a subsequent call to
\funcname{krb5_kt_next_entry}.

May return error code if \funcparam{cursor} is invalid.




\subsection{Operating-system specific functions}
The operating-system specific functions provide an interface between the
other parts of the \libname{libkrb5.a} libraries and the operating system.

Beware! Any of these are allowed to be implemented as macros.

The following global symbols are provided in \libname{libos.a}.  If you
wish to substitute for any of them, you must substitute for all of them
(they are all declared and initialized in the same object file):
\begin{itemize}
% These come from src/lib/osconfig.c
\item extern char *\globalname{krb5_config_file}: name of configuration file
\item extern char *\globalname{krb5_trans_file}: name of hostname/realm
name translation file
\item extern char *\globalname{krb5_defkeyname}: default name of key
table file
\item extern char *\globalname{krb5_lname_file}: name of aname/lname
translation database
\item extern int \globalname{krb5_max_dgram_size}: maximum allowable
datagram size
\item extern int \globalname{krb5_max_skdc_timeout}: maximum
per-message KDC reply timeout
\item extern int \globalname{krb5_skdc_timeout_shift}: shift factor
(bits) to exponentially back-off the KDC timeouts
\item extern int \globalname{krb5_skdc_timeout_1}: initial KDC timeout
\item extern char *\globalname{krb5_kdc_udp_portname}: name of KDC UDP port
\item extern char *\globalname{krb5_default_pwd_prompt1}: first prompt
for password reading.
\item extern char *\globalname{krb5_default_pwd_prompt2}: second prompt

\end{itemize}

\begin{funcdecl}{krb5_read_password}{krb5_error_code}{\funcin}
\funcarg{char *}{prompt}
\funcarg{char *}{prompt2}
\funcout
\funcarg{char *}{return_pwd}
\funcinout
\funcarg{int *}{size_return}
\end{funcdecl}

Read a password from the keyboard.  The first \funcparam{*size_return}
bytes of the password entered are returned in \funcparam{return_pwd}.
If fewer than \funcparam{*size_return} bytes are typed as a password,
the remainder of \funcparam{return_pwd} is zeroed.  Upon success, the
total number of bytes filled in is stored in \funcparam{*size_return}.

\funcparam{prompt} is used as the prompt for the first reading of a password.
It is printed to the terminal, and then a password is read from the
keyboard.  No newline or spaces are emitted between the prompt and the
cursor, unless the newline/space is included in the prompt.

If \funcparam{prompt2} is a null pointer, then the password is read
once.  If \funcparam{prompt2} is set, then it is used as a prompt to
read another password in the same manner as described for
\funcparam{prompt}.  After the second password is read, the two
passwords are compared, and an error is returned if they are not
identical.

Echoing is turned off when the password is read.

If there is an error in reading or verifying the password, an error code
is returned; else zero is returned.

\begin{funcdecl}{krb5_lock_file}{krb5_error_code}{\funcvoid}
\funcarg{FILE *}{filep}
\funcarg{char *}{pathname}
\funcarg{int}{mode}
\end{funcdecl}

Attempts to lock the file in the given \funcparam{mode}; returns 0 for a
successful lock, or an error code otherwise.

The caller should arrange that both \funcparam{filep} and
\funcparam{pathname} refer to the same
file.  The implementation may use whichever is more convenient.

Modes are given in {\tt <krb5/libos.h>}


\begin{funcdecl}{krb5_unlock_file}{krb5_error_code}{\funcvoid}
\funcarg{FILE *}{filep}
\funcarg{char *}{pathname}
\end{funcdecl}

Attempts to (completely) unlock the file.  Returns 0 if successful,
or an error code otherwise.

The caller should arrange that both \funcparam{filep} and
\funcparam{pathname} refer to the same file.  The implementation may
use whichever is more convenient.

\begin{funcdecl}{krb5_timeofday}{krb5_error_code}{\funcout}
\funcarg{krb5_int32 *}{timeret}
\end{funcdecl}

Retrieves the system time of day, in seconds since the local system's
epoch.
[The ASN.1 encoding routines must convert this to the standard ASN.1
encoding as needed]

\begin{funcdecl}{krb5_ms_timeofday}{krb5_error_code}{\funcout}
\funcarg{int32 *}{seconds}
\funcarg{int16 *}{milliseconds}
\end{funcdecl}

Retrieves the system time of day, in seconds since the local system's
epoch.
[The ASN.1 encoding routines must convert this to the standard ASN.1
encoding as needed]

The seconds portion is returned in \funcparam{*seconds}, the
milliseconds portion in \funcparam{*milliseconds}.

\begin{funcdecl}{krb5_net_read}{int}{\funcin}
\funcarg{int}{fd}
\funcout
\funcarg{char *}{buf}
\funcin
\funcarg{int}{len}
\end{funcdecl}

Like read(2), but guarantees that it reads as much as was requested
or returns -1 and sets errno.

(make sure your sender will send all the stuff you are looking for!)
Only useful on stream sockets and pipes.

\begin{funcdecl}{krb5_net_write}{int}{\funcin}
\funcarg{int}{fd}
\funcarg{const char *}{buf}
\funcarg{int}{len}
\end{funcdecl}

Like write(2), but guarantees that it writes as much as was requested
or returns -1 and sets errno.

(make sure your sender will send all the stuff you are looking for!)
Only useful on stream sockets and pipes.

\begin{funcdecl}{krb5_os_localaddr}{krb5_error_code}{\funcout}
\funcarg{krb5_address ***}{addr}
\end{funcdecl}

Return all the protocol addresses of this host.

Compile-time configuration flags will indicate which protocol family
addresses might be returned.
\funcparam{*addr} is filled in to point to an array of address pointers,
terminated by a null pointer.  All the storage pointed to is allocated
and should be freed by the caller with \funcname{krb5_free_address}
when no longer needed.


\begin{funcdecl}{krb5_sendto_kdc}{krb5_error_code}{\funcin}
\funcarg{krb5_data *}{send}
\funcarg{krb5_data *}{realm}
\funcout
\funcarg{krb5_data *}{receive}
\end{funcdecl}

Send the message \funcparam{send} to a KDC for realm \funcparam{realm} and
return the response (if any) in \funcparam{receive}.

If the message is sent and a response is received, 0 is returned,
otherwise an error code is returned.

The storage for \funcparam{receive} is allocated and should be freed by
the caller when finished.

\begin{funcdecl}{krb5_get_krbhst}{krb5_error_code}{\funcin}
\funcarg{krb5_data *}{realm}
\funcout
\funcarg{char ***}{hostlist}
\end{funcdecl}

Figures out the Kerberos server names for the given \funcparam{realm},
filling in
\funcparam{hostlist} with a
pointer to an argv[] style list of names, terminated with a null
pointer.
 
If \funcparam{realm} is unknown, the filled-in pointer is set to NULL.

The pointer array and strings pointed to are all in allocated storage,
and should be freed by the caller when finished.

Returns system errors.

\begin{funcdecl}{krb5_free_krbhst}{krb5_error_code}{\funcin}
\funcarg{char **}{hostlist}
\end{funcdecl}

Frees the storage taken by a host list returned by \funcname{krb5_get_krbhst}.

\begin{funcdecl}{krb5_aname_to_localname}{krb5_error_code}{\funcin}
\funcarg{krb5_principal}{aname}
\funcarg{int}{lnsize}
\funcout
\funcarg{char *}{lname}
\end{funcdecl}

Converts a principal name \funcparam{aname} to a local name suitable for use by
programs wishing a translation to an environment-specific name (e.g.
user account name).

\funcparam{lnsize} specifies the maximum length name that is to be filled into
\funcparam{lname}.
The translation will be null terminated in all non-error returns.

Returns system errors.

\begin{funcdecl}{krb5_get_default_realm}{krb5_error_code}{\funcin}
\funcarg{int}{lnsize}
\funcout
\funcarg{char *}{lrealm}
\end{funcdecl}

Retrieves the default realm to be used if no user-specified realm is
available (e.g. to interpret a user-typed principal name with the
realm omitted for convenience).

\funcparam{lnsize} specifies the maximum length name that is to be filled into
\funcparam{lrealm}.

Returns system errors.

\begin{funcdecl}{krb5_get_host_realm}{krb5_error_code}{\funcin}
\funcarg{char *}{host}
\funcout
\funcarg{char ***}{realmlist}
\end{funcdecl}

Figures out the Kerberos realm names for \funcparam{host}, filling in
\funcparam{realmlist} with a
pointer to an argv[] style list of names, terminated with a null pointer.
 
If \funcparam{host} is NULL, the local host's realms are determined.

If there are no known realms for the host, the filled-in pointer is set
to NULL.

The pointer array and strings pointed to are all in allocated storage,
and should be freed by the caller when finished.

Returns system errors.

\begin{funcdecl}{krb5_free_host_realm}{krb5_error_code}{\funcin}
\funcarg{char **}{realmlist}
\end{funcdecl}

Frees the storage taken by a \funcparam{realmlist} returned by
\funcname{krb5_get_local_realm}.

\begin{funcdecl}{krb5_kuserok}{krb5_boolean}{\funcin}
\funcarg{krb5_principal}{principal}
\funcarg{const char *}{luser}
\end{funcdecl}

Given a Kerberos principal \funcparam{principal}, and a local username
\funcparam{luser},
determine whether user is authorized to login to the account \funcparam{luser}.
Returns TRUE if authorized, FALSE if not authorized.

\begin{funcdecl}{krb5_random_confounder}{krb5_confounder}{\funcvoid}
\end{funcdecl}

Generate a random confounder.


\section{Principal database functions}

\input{kdb.tex}

\section{Encryption system interface}
\input{encrypt.tex}

\section{Checksum interface}
\input{cksum.tex}

\section{CRC-32 checksum functions}
\input{crc-32.tex}

\appendix
\cleardoublepage
\input{\jobname.ind}
\end{document}
