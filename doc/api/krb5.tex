The main functions deal with the nitty-gritty details: verifying
tickets, creating authenticators, and the like.

\begin{funcdecl}{krb5_encode_kdc_rep}{krb5_error_code}{\funcin}
\funcarg{krb5_msgtype}{type}
\funcarg{krb5_enc_kdc_rep_part *}{encpart}
\funcarg{krb5_keyblock *}{client_key}
\funcinout
\funcarg{krb5_kdc_rep *}{dec_rep}
\funcout
\funcarg{krb5_data *}{enc_rep}
\end{funcdecl}

Takes KDC rep parts in \funcparam{*rep} and \funcparam{*encpart}, and
formats it into \funcparam{*enc_rep}, using message type \funcparam{type}
and encryption key \funcparam{client_key} and encryption type
\funcparam{dec_rep{\ptsto}etype}.

\funcparam{enc_rep{\ptsto}data} will point to  allocated storage upon
non-error return; the caller should free it when finished.

Returns system errors.

\begin{funcdecl}{krb5_decode_kdc_rep}{krb5_error_code}{\funcin}
\funcarg{krb5_data *}{enc_rep}
\funcarg{krb5_keyblock *}{key}
\funcarg{krb5_enctype}{etype}
\funcout
\funcarg{krb5_kdc_rep **}{dec_rep}
\end{funcdecl}

Takes a KDC_REP message and decrypts encrypted part using
\funcparam{etype} and \funcparam{*key}, putting result in \funcparam{*rep}.
The pointers in \funcparam{dec_rep}
are all set to allocated storage which should be freed by the caller
when finished with the response (by using \funcname{krb5_free_kdc_rep}).


If the response isn't a KDC_REP (tgs or as), it returns an error from
the decoding routines (usually ISODE_50_LOCAL_ERR_BADDECODE).

Returns errors from encryption routines, system errors.

\begin{funcdecl}{krb5_kdc_rep_decrypt_proc}{\funcin}
\funcarg{krb5_keyblock *}{key}
\funcarg{krb5_pointer}{decryptarg}
\funcinout
\funcarg{krb5_kdc_rep *}{dec_rep}
\end{funcdecl}
Decrypt the encrypted portion of \funcparam{dec_rep}, using the
encryption key \funcparam{key}.

The result is in allocated storage pointed to by
\funcparam{dec_rep{\ptsto}enc_part2}, unless some error occurs.

\begin{funcdecl}{krb5_encode_ticket}{krb5_error_code}{\funcin}
\funcarg{krb5_ticket *}{dec_ticket}
\funcout
\funcarg{krb5_data **}{enc_ticket}
\end{funcdecl}

Takes \funcparam{dec_ticket} (with associated encrypted part
\funcparam{dec_ticket{\ptsto}enc_part}),
and encodes for transmission, placing result in \funcparam{*enc_ticket}.
The string \funcparam{*enc_ticket} will be allocated before formatting.

Returns errors from encryption routines, system errors.

\begin{funcdecl}{krb5_decode_ticket}{krb5_error_code}{\funcin}
\funcarg{krb5_data *}{enc_ticket}
\funcout
\funcarg{krb5_ticket **}{dec_ticket}
\end{funcdecl}

Decodes formatted ticket \funcparam{enc_ticket},
filling in \funcparam{*dec_ticket} with a pointer to the results.
\funcparam{*dec_ticket} is set to allocated storage which should be
freed by the caller (by using \funcname{krb5_free_ticket}) when finished with
the ticket.

Returns system errors.


\begin{funcdecl}{krb5_encrypt_tkt_part}{krb5_error_code}{ \funcin}
\funcarg{krb5_keyblock *}{srv_key}
\funcinout
\funcarg{krb5_ticket *}{dec_ticket}
\end{funcdecl}

Takes unencrypted \funcparam{dec_ticket} and
\funcparam{dec_ticket{\ptsto}enc_part2}, encrypts with
\funcparam{dec_ticket{\ptsto}etype}
using \funcparam{srv_key}, and places result in
\funcparam{dec_ticket{\ptsto}enc_part}.
The string \funcparam{dec_ticket{\ptsto}enc_part} will be allocated
before formatting.

Returns errors from encryption routines, system errors

\funcparam{enc_part{\ptsto}data} is allocated and filled in with
encrypted stuff.

\begin{funcdecl}{krb5_decrypt_tkt_part}{krb5_error_code}{\funcin}
\funcarg{krb5_keyblock *}{srv_key}
\funcinout
\funcarg{krb5_ticket *}{dec_ticket}
\end{funcdecl}

Takes encrypted \funcparam{dec_ticket{\ptsto}enc_part}, encrypts with
\funcparam{dec_ticket{\ptsto}etype}
using \funcparam{srv_key}, and places result in
\funcparam{dec_ticket{\ptsto}enc_part2}.  The storage of
\funcparam{dec_ticket{\ptsto}enc_part2} will be allocated before return.

Returns errors from encryption routines, system errors

\begin{funcdecl}{krb5_send_tgs}{krb5_error_code}{\funcin}
\funcarg{krb5_flags}{options}
\funcarg{krb5_ticket_times *}{timestruct}
\funcarg{krb5_enctype}{etype}
\funcarg{krb5_cksumtype}{sumtype}
\funcarg{krb5_principal}{sname}
\funcarg{krb5_address **}{addrs}
\funcarg{krb5_authdata **}{authorization_data}
\funcarg{krb5_data *}{second_ticket}
\funcinout
\funcarg{krb5_creds *}{usecred}
\funcout
\funcarg{krb5_response *}{rep}
\end{funcdecl}

Sends a request to the TGS and waits for a response.
\funcparam{options} is used for the options in the KRB_TGS_REQ.
\funcparam{timestruct} values are used for from, till, and rtime in the
KRB_TGS_REQ.
\funcparam{etype} is used for etype in the KRB_TGS_REQ.
\funcparam{sumtype} is used for the checksum in the AP_REQ in the KRB_TGS_REQ
\funcparam{sname} is used for sname in the KRB_TGS_REQ.
\funcparam{addrs}, if non-NULL, is used for addresses in the KRB_TGS_REQ.
\funcparam{authorization_dat}, if non-NULL, is used for authorization_dat in the KRB_TGS_REQ.
\funcparam{second_ticket}, if required by options, is used for the 2nd
ticket in the KRB_TGS_REQ.
\funcparam{usecred} is used for the ticket and session key in the KRB_AP_REQ header in the KRB_TGS_REQ.

The KDC realm is extracted from \funcparam{usecred{\ptsto}server}'s realm.

The response is placed into \funcparam{*rep}.
\funcparam{rep{\ptsto}response.data} is set to point at allocated storage
which should be freed by the caller when finished.

Returns system errors.

\begin{funcdecl}{krb5_get_cred_from_kdc}{krb5_error_code}{\funcin}
\funcarg{krb5_ccache}{ccache}
\funcinout
\funcarg{krb5_creds *}{creds}
\funcout			
\funcparam{krb5_creds ***}{tgts }
\end{funcdecl}

Retrieve credentials for principal \funcparam{creds{\ptsto}client},
server \funcparam{creds{\ptsto}server},
ticket flags \funcparam{creds{\ptsto}ticket_flags}, possibly
\funcparam{creds{\ptsto}second_ticket} if needed by the ticket flags.

\funcparam{ccache} is used to fetch initial TGT's to start the authentication
path to the server.

Credentials are requested from the KDC for the server's realm.  Any
TGT credentials obtained in the process of contacting the KDC are
returned in an array of credentials; \funcparam{tgts} is filled in to
point to an array of pointers to credential structures (if no TGT's were
used, the pointer is zeroed).  TGT's may be returned even if no useful
end ticket was obtained.

The returned credentials are NOT cached.

If credentials are obtained, \funcparam{creds} is filled in with the results;
\funcparam{creds{\ptsto}ticket} and
\funcparam{creds{\ptsto}keyblock{\ptsto}key} are set to allocated storage,
which should be freed by the caller when finished.

Returns errors, system errors.


\begin{funcdecl}{krb5_free_tgt_creds}{void}{\funcin}
\funcarg{krb5_creds **}{tgts}
\end{funcdecl}

Frees the TGT credentials \funcparam{tgts} returned by
\funcname{krb5_get_cred_from_kdc}.

\begin{funcdecl}{krb5_get_credentials}{krb5_error_code}{\funcin}
\funcarg{krb5_flags}{options}
\funcarg{krb5_ccache}{ccache}
\funcinout
\funcarg{krb5_creds *}{creds}
\end{funcdecl}

Attempts to use the credentials cache \funcparam{ccache} or a TGS
exchange to get an additional ticket for the client identified by
\funcparam{creds{\ptsto}client}, the server identified by
\funcparam{creds{\ptsto}server}, with options \funcparam{options},
expiration date specified in \funcparam{creds{\ptsto}times.endtime} (0
means as long as possible), session key type specified in
\funcparam{creds{\ptsto}keyblock.keytype} (if non-zero).

Any returned ticket and intermediate ticket-granting tickets are
stored in \funcparam{ccache}.

Returns errors from encryption routines, system errors.

\begin{funcdecl}{krb5_get_in_tkt}{krb5_error_code}{\funcin}
\funcarg{krb5_flags}{options}
\funcarg{krb5_address **}{addrs}
\funcarg{krb5_enctype}{etype}
\funcarg{krb5_keytype}{keytype}
\funcfuncarg{krb5_error_code}{(*key_proc)}
	\funcarg{krb5_keytype}{type}
	\funcarg{krb5_keyblock **}{key}
	\funcarg{krb5_pointer}{keyseed}
\funcendfuncarg
\funcarg{krb5_pointer}{keyseed}
\funcfuncarg{krb5_error_code}{(*decrypt_proc)}
	\funcarg{krb5_keyblock *}{key}
	\funcarg{krb5_pointer}{decryptarg}
	\funcarg{krb5_kdc_rep *}{dec_rep}
\funcendfuncarg
\funcarg{krb5_pointer}{decryptarg}
\funcinout
\funcarg{krb5_creds *}{creds}
\funcarg{krb5_ccache}{ccache}
\end{funcdecl}

All-purpose initial ticket routine, usually called via
\funcname{krb5_get_in_tkt_with_password} or
\funcname{krb5_get_in_tkt_with_skey}.

Attempts to get an initial ticket for \funcparam{creds{\ptsto}client} to use server
\funcparam{creds{\ptsto}server}, (realm is taken from
\funcparam{creds{\ptsto}client}), with options 
\funcparam{options}, requesting encryption type \funcparam{etype}, and using
\funcparam{creds{\ptsto}times.starttime},  \funcparam{creds{\ptsto}times.endtime},
\funcparam{creds{\ptsto}times.renew_till}
as from, till, and rtime.  \funcparam{creds{\ptsto}times.renew_till} is
ignored unless the RENEWABLE option is requested.

\funcparam{key_proc} is called to fill in the key to be used for decryption.
\funcparam{keyseed} is passed on to \funcparam{key_proc}.

\funcparam{decrypt_proc} is called to perform the decryption of the
response (the encrypted part is in \funcparam{dec_rep{\ptsto}enc_part}; the
decrypted part should be allocated and filled into
\funcparam{dec_rep{\ptsto}enc_part2}.
\funcparam{decryptarg} is passed on to \funcparam{decrypt_proc}.

If \funcparam{addrs} is non-NULL, it is used for the addresses
requested.  If it is null, the system standard addresses are used.

A succesful call will place the ticket in the credentials cache
\funcparam{ccache} and fill in \funcparam{creds} with the ticket
information used/returned.

Returns system errors, encryption errors.

\begin{funcdecl}{krb5_get_in_tkt_with_password}{krb5_error_code}{\funcin}
\funcarg{krb5_flags}{options}
\funcarg{krb5_address **}{addrs}
\funcarg{krb5_enctype}{etype}
\funcarg{krb5_keytype}{keytype}
\funcarg{char *}{password}
\funcarg{krb5_ccache}{ccache}
\funcinout
\funcarg{krb5_creds *}{creds}
\end{funcdecl}


Attempts to get an initial ticket for \funcparam{creds{\ptsto}client} to use server
\funcparam{creds{\ptsto}server}, (realm is taken from
\funcparam{creds{\ptsto}client}), with options 
\funcparam{options}, requesting encryption type \funcparam{etype}, and using
\funcparam{creds{\ptsto}times.starttime},
\funcparam{creds{\ptsto}times.endtime},
\funcparam{creds{\ptsto}times.renew_till}
as from, till, and rtime.  \funcparam{creds{\ptsto}times.renew_till} is
ignored unless the RENEWABLE option is requested.

If \funcparam{addrs} is non-NULL, it is used for the addresses
requested.  If it is null, the system standard addresses are used.

If \funcparam{password} is non-NULL, it is converted using the
cryptosystem entry point for a string conversion routine, seeded with
the client's principal name.  If \funcparam{password} is passed as NULL,
the password is read from the terminal, and then converted into a key.

A succesful call will place the ticket in the credentials cache
\funcparam{ccache}.

Returns system errors, encryption errors.

\begin{funcdecl}{krb5_get_in_tkt_with_skey}{krb5_error_code}{\funcin}
\funcarg{krb5_flags}{options}
\funcarg{krb5_address **}{addrs}
\funcarg{krb5_enctype}{etype}
\funcarg{krb5_keyblock *}{key}
\funcarg{krb5_ccache}{ccache}
\funcinout
\funcarg{krb5_creds *}{creds}
\end{funcdecl}
Similar to \funcname{krb5_get_in_tkt_with_password}.

Attempts to get an initial ticket for \funcparam{creds{\ptsto}client} to use server
\funcparam{creds{\ptsto}server}, (realm is taken from
\funcparam{creds{\ptsto}client}), with options \funcparam{options}, requesting
encryption type \funcparam{etype}, and using 
\funcparam{creds{\ptsto}times.starttime}, \funcparam{creds{\ptsto}times.endtime},
\funcparam{creds{\ptsto}times.renew_till} as from, till, and rtime.
\funcparam{creds{\ptsto}times.renew_till} is ignored unless the
RENEWABLE option is requested.

If \funcparam{addrs} is non-NULL, it is used for the addresses
requested.  If it is null, the system standard addresses are used.

If \funcparam{keyblock} is NULL, an appropriate key for
\funcparam{creds{\ptsto}client} is retrieved from the system key store (e.g.
\filename{/etc/v5srvtab}).  If \funcparam{keyblock} is non-NULL, it is
used as the decryption key.

A succesful call will place the ticket in the credentials cache
\funcparam{ccache}.

Returns system errors, encryption errors.

\begin{funcdecl}{krb5_mk_req}{krb5_error_code}{\funcin}
\funcarg{krb5_principal}{server}
\funcarg{krb5_flags}{ap_req_options}
\funcarg{krb5_checksum *}{checksum}
\funcarg{krb5_ccache}{ccache}
\funcout
\funcarg{krb5_data *}{outbuf}
\end{funcdecl}

Formats a KRB_AP_REQ message into \funcparam{outbuf}.

\funcparam{server} specifies the principal of the server to receive the
message; if credentials are not present in the credentials cache
\funcparam{ccache} for this server, the TGS request with default
parameters is used in an attempt to obtain such credentials, and they
are stored in \funcparam{ccache}.

\funcparam{ap_req_options} specifies the KRB_AP_REQ options desired.

\funcparam{checksum} specifies the checksum to be used in the authenticator.

The \funcparam{outbuf} buffer storage is allocated, and should be freed
by the caller when finished.

Returns system errors.


\begin{funcdecl}{krb5_mk_req_extended}{krb5_error_code}{\funcin}
\funcarg{krb5_flags}{ap_req_options}
\funcarg{krb5_checksum *}{checksum}
\funcarg{krb5_ticket_times *}{times}
\funcarg{krb5_flags}{kdc_options}
\funcarg{krb5_ccache}{ccache}
\funcinout
\funcarg{krb5_creds *}{creds}
\funcout
\funcarg{krb5_data *}{outbuf}
\end{funcdecl}

Formats a KRB_AP_REQ message into \funcparam{outbuf}, with more complete
options than \funcname{krb_mk_req}.

\funcparam{outbuf}, \funcparam{ap_req_options}, \funcparam{checksum},
and \funcparam{ccache} are used in the same fashion as for
\funcname{krb5_mk_req}.

\funcparam{creds} is used to supply the credentials (ticket and session
key) needed to form the request.

If \funcparam{creds{\ptsto}ticket} has no data (length == 0), then a
ticket is obtained from either \funcparam{ccache} or the TGS, passing
\funcparam{creds} to \funcname{krb5_get_credentials}. 
\funcparam{kdc_options} specifies the options requested for the ticket
to be used. If a ticket with appropriate flags is not found in
\funcparam{ccache}, then these options are passed on in a request to an
appropriate KDC.

\funcparam{ap_req_options} specifies the KRB_AP_REQ options desired.

If \funcparam{ap_req_options} specifies AP_OPTS_USE_SESSION_KEY, then
\funcparam{creds{\ptsto}ticket} must contain the appropriate
ENC-TKT-IN-SKEY ticket.

\funcparam{checksum} specifies the checksum to be used in the authenticator.

The \funcparam{outbuf} buffer storage is allocated, and should be freed
by the caller when finished.

Returns system errors.


\begin{funcdecl}{krb5_rd_req_simple}{krb5_error_code}{\funcin}
\funcarg{const krb5_data *}{inbuf}
\funcarg{const krb5_principal}{server}
\funcarg{constkrb5_address *}{sender_addr}
\funcout
\funcarg{krb5_tkt_authent *}{authdat}
\end{funcdecl}

Parses a KRB_AP_REQ message, returning its contents.

\funcparam{server} specifies the expected server's name for the ticket.

\funcparam{sender_addr} specifies the address(es) expected to be present
in the ticket.

A replay cache name derived from the first component of the service name
is used.

The default key store is consulted to find the service key.

\funcparam{authdat{\ptsto}ticket} and
\funcparam{authdat{\ptsto}authenticator} are set to allocated storage
structures; the caller should free them when finished.

Returns system errors, encryption errors, replay errors.


\begin{funcdecl}{krb5_rd_req}{krb5_error_code}{\funcin}
\funcarg{krb5_data *}{inbuf}
\funcarg{krb5_principal}{server}
\funcarg{krb5_address *}{sender_addr}
\funcarg{krb5_pointer}{fetchfrom}
\funcfuncarg{krb5_error_code}{(*keyproc)}
\funcarg{krb5_pointer}{keyprocarg}
\funcarg{krb5_principal}{principal}
\funcarg{krb5_kvno}{vno, krb5_keyblock **key}
\funcendfuncarg
\funcarg{krb5_pointer}{keyprocarg}
\funcinout
\funcarg{krb5_rcache}{rcache}
\funcout
\funcarg{krb5_tkt_authent *}{authdat}
\end{funcdecl}


Parses a KRB_AP_REQ message, returning its contents.

\funcparam{server} specifies the expected server's name for the ticket.

\funcparam{sender_addr} specifies the address(es) expected to be present
in the ticket.

\funcparam{rcache} specifies a replay detection cache used to store
authenticators and server names.

\funcparam{keyproc} specifies a procedure to generate a decryption key for the
ticket.  If \funcparam{keyproc} is non-NULL, \funcparam{keyprocarg} is
passed to it, and the result used as a decryption key. If
\funcparam{keyproc} is NULL, then \funcparam{fetchfrom} is checked; if
it is non-NULL, it specifies a parameter name from which to retrieve the
decryption key.  If \funcparam{fetchfrom} is NULL, then the default key
store is consulted.

\funcparam{authdat{\ptsto}ticket} and
\funcparam{authdat{\ptsto}authenticator} are set to allocated storage
structures; the caller should free them when finished.

Returns system errors, encryption errors, replay errors.

\begin{funcdecl}{krb5_rd_req_decoded}{krb5_error_code}{\funcin}
\funcarg{krb5_ap_req *}{req}
\funcarg{krb5_principal}{server}
\funcarg{krb5_address *}{sender_addr}
\funcarg{krb5_pointer}{fetchfrom}
\funcfuncarg{krb5_error_code}{(*keyproc)}
\funcarg{krb5_pointer}{keyprocarg}
\funcarg{krb5_principal}{principal}
\funcarg{krb5_kvno}{vno}
\funcarg{krb5_keyblock **}{key}
\funcendfuncarg
\funcarg{krb5_pointer}{keyprocarg}
\funcarg{krb5_rcache}{rcache}
\funcout
\funcarg{krb5_tkt_authent *}{authdat}
\end{funcdecl}

Essentially the same as \funcname{krb_rd_req}, but uses a decoded AP_REQ
as the input rather than an encoded input.

\begin{funcdecl}{krb5_mk_error}{krb5_error_code}{\funcin}
\funcarg{krb5_error *}{dec_err}
\funcout
\funcarg{krb5_data *}{enc_err}
\end{funcdecl}

Formats the error structure \funcparam{*dec_err} into an error buffer
\funcparam{*enc_err}.

The error buffer storage is allocated, and should be freed by the
caller when finished.

Returns system errors.

\begin{funcdecl}{krb5_rd_error}{krb5_error_code}{\funcin}
\funcarg{krb5_data *}{enc_errbuf}
\funcout
\funcarg{krb5_error *}{dec_error}
\end{funcdecl}

Parses an error message from \funcparam{enc_errbuf} and fills in the
contents of \funcparam{dec_error}.

Upon return \funcparam{dec_error{\ptsto}client},
\funcparam{dec_error{\ptsto}server}, and
\funcparam{dec_error{\ptsto}text}, if non-NULL, point to allocated
storage which the caller should free when finished.

Returns system errors.

\begin{funcdecl}{krb5_mk_safe}{krb5_error_code}{\funcin}
\funcarg{krb5_data *}{userdata}
\funcarg{krb5_cksumtype}{sumtype}
\funcarg{krb5_keyblock *}{key,}
\funcarg{krb5_fulladdr *}{sender_addr}
\funcarg{krb5_fulladdr *}{recv_addr}
\funcout
\funcarg{krb5_data *}{outbuf}
\end{funcdecl}

Formats a KRB_SAFE message into \funcparam{outbuf}.

\funcparam{userdata} is formatted as the user data in the message.
\funcparam{sumtype} specifies the encryption type; \funcparam{key}
specifies the key which might be used to seed the checksum;
\funcparam{sender_addr} and \funcparam{recv_addr} specify the full
addresses (host and port) of the sender and receiver.  The host portion
of \funcparam{sender_addr} is used to form the addresses used in the
KRB_SAFE message.

The \funcparam{outbuf} buffer storage is allocated, and should be freed by the
caller when finished.

Returns system errors.

\begin{funcdecl}{krb5_rd_safe}{krb5_error_code}{\funcin}
\funcarg{krb5_data *}{inbuf}
\funcarg{krb5_keyblock *}{key}
\funcarg{krb5_fulladdr *}{sender_addr}
\funcarg{krb5_fulladdr *}{recv_addr}
\funcout
\funcarg{krb5_data *}{outbuf}
\end{funcdecl}

Parses a KRB_SAFE message from \funcparam{inbuf}, placing the
integrity-protected user data in \funcparam{*outbuf}.

\funcparam{key} specifies the key to be used for decryption of the message.
 
\funcparam{sender_addr} and \funcparam{recv_addr} specify the full
addresses (host and port) of the sender and receiver.

\funcparam{outbuf} points to allocated storage which the caller should
free when finished.

Returns system errors, integrity errors.

\begin{funcdecl}{krb5_mk_priv}{krb5_error_code}{\funcin}
\funcarg{krb5_data *}{userdata}
\funcarg{krb5_enctype}{etype}
\funcarg{krb5_keyblock *}{key}
\funcarg{krb5_fulladdr *}{sender_addr}
\funcarg{krb5_fulladdr *}{recv_addr}
\funcout
\funcarg{krb5_data *}{outbuf}
\end{funcdecl}

Formats a KRB_PRIV message into \funcparam{outbuf}.

\funcparam{userdata} is formatted as the user data in the message.
\funcparam{etype} specifies the encryption type; \funcparam{key}
specifies the key for the encryption; \funcparam{sender_addr} and
\funcparam{recv_addr} specify the full addresses (host and port) of the
sender and receiver.

The \funcparam{outbuf} buffer storage is allocated, and should be freed by the
caller when finished.

Returns system errors.

\begin{funcdecl}{krb5_rd_priv}{krb5_error_code}{\funcin}
\funcarg{krb5_data *}{inbuf}
\funcarg{krb5_keyblock *}{key}
\funcarg{krb5_fulladdr *}{sender_addr}
\funcarg{krb5_fulladdr *}{recv_addr}
\funcout
\funcarg{krb5_data *}{outbuf}
\end{funcdecl}

Parses a KRB_PRIV message from \funcparam{inbuf}, placing the confidential user
data in \funcparam{*outbuf}.

\funcparam{key} specifies the key to be used for decryption of the message.
 
\funcparam{sender_addr} and \funcparam{recv_addr} specify the full
addresses (host and port) of the sender and receiver.

\funcparam{outbuf} points to allocated storage which the caller should
free when finished.

Returns system errors, integrity errors.

\begin{funcdecl}{krb5_parse_name}{krb5_error_code}{\funcin}
\funcarg{char *}{name}
\funcout
\funcarg{krb5_principal *}{principal}
\end{funcdecl}

Converts a single-string representation \funcparam{name} of the
principal name to the multi-part principal format used in the protocols.

\funcparam{*principal} will point to allocated storage which should be freed by
the caller (using \funcname{krb5_free_principal}) after use.

Returns system errors XXX.

\begin{funcdecl}{krb5_unparse_name}{krb5_error_code}{\funcin}
\funcarg{krb5_principal}{principal}
\funcout
\funcarg{char **}{name}
\end{funcdecl}

Converts the multi-part principal name \funcparam{principal} from the
format used in the protocols to a single-string representation of the name.

\funcparam{*name} points to allocated storage and should be freed by the caller
when finished.

Returns system errors XXX.

\begin{funcdecl}{krb5_address_search}{krb5_boolean}{\funcin}
\funcarg{krb5_address *}{addr}
\funcarg{krb5_address **}{addrlist}
\end{funcdecl}

If \funcparam{addr} is listed in \funcparam{addrlist}, or
\funcparam{addrlist} is null, return TRUE.  If not listed, return FALSE.

\begin{funcdecl}{krb5_address_compare}{krb5_boolean}{\funcin}
\funcarg{krb5_address *}{addr1}
\funcarg{krb5_address *}{addr2}
\end{funcdecl}

If the two addresses are the same, return TRUE, else return FALSE.

\begin{funcdecl}{krb5_principal_compare}{krb5_boolean}{\funcin}
\funcarg{krb5_principal}{p1}
\funcarg{krb5_principal}{p2}
\end{funcdecl}

If the two principals are the same, return TRUE, else return FALSE.

\begin{funcdecl}{krb5_fulladdr_order}{int}{\funcin}
\funcarg{krb5_fulladdr *}{addr1}
\funcarg{krb5_fulladdr *}{addr2}
\end{funcdecl}

Return an ordering on the two full addresses:  0 if the same,
$< 0$ if first is less than 2nd, $> 0$ if first is greater than 2nd.


\begin{funcdecl}{krb5_copy_keyblock}{krb5_error_code}{\funcin}
\funcarg{krb5_keyblock *}{from}
\funcout
\funcarg{krb5_keyblock *}{to}
\end{funcdecl}

Copy a keyblock from \funcparam{from} to \funcparam{to}, including
allocated storage.

\begin{funcdecl}{krb5_copy_creds}{krb5_error_code}{\funcin}
\funcarg{krb5_creds *}{incred}
\funcout
\funcarg{krb5_creds **}{outcred}
\end{funcdecl}

Copy a credentials structure, filling in \funcparam{*outcred} to point
to the newly allocated copy, which should be freed with
\funcname{krb5_free_creds}.

\begin{funcdecl}{krb5_copy_data}{krb5_error_code}{\funcin}
\funcarg{krb5_data *}{indata}
\funcout
\funcarg{krb5_data **}{outdata}
\end{funcdecl}

Copy a data strucutre, filling in \funcparam{*outdata} to point to the
newly allocated copy, which should be freed with \funcname{krb5_free_data}.

\begin{funcdecl}{krb5_copy_principal}{krb5_error_code}{\funcin}
\funcarg{krb5_principal}{inprinc}
\funcout
\funcarg{krb5_principal *}{outprinc}
\end{funcdecl}
Copy a principal structure, filling in \funcparam{*outprinc} to point to
the newly allocated copy, which should be freed with
\funcname{krb5_free_principal}.

