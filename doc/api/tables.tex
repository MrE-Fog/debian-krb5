The following is a list of options which can be passed to the Kerberos
server (also known as the Key Distribution Center or KDC).  These
options affect what sort of tickets the KDC will return to the
application program.  The KDC options can be passed to
\funcname{krb5_get_in_tkt}, \funcname{krb5_get_in_tkt_with_password},
\funcname{krb5_get_in_tkt_with_skey}, and \funcname{krb5_send_tgs}. 


\begin{center}
\begin{tabular}{llc}
\multicolumn{1}{c}{Symbol}&\multicolumn{1}{c}{RFC}& Valid for \\
&\multicolumn{1}{c}{section}&get_in_tkt? \\ \hline
KDC_OPT_FORWARDABLE	& 2.6	& yes		\\
KDC_OPT_FORWARDED	& 2.6	&		\\
KDC_OPT_PROXIABLE	& 2.5	& yes		\\
KDC_OPT_PROXY		& 2.5	&		\\
KDC_OPT_ALLOW_POSTDATE	& 2.4	& yes		\\
KDC_OPT_POSTDATED	& 2.4	& yes		\\
KDC_OPT_RENEWABLE	& 2.3	& yes		\\
KDC_OPT_RENEWABLE_OK	& 2.7	& yes		\\
KDC_OPT_ENC_TKT_IN_SKEY	& 2.7	&		\\
KDC_OPT_RENEW		& 2.3	&		\\
KDC_OPT_VALIDATE	& 2.2	&		\\
\end{tabular}
\end{center}
\label{KDCOptions}

The following is a list of preauthentication methods which are supported
by Kerberos.  Most preauthentication methods are used by
\funcname{krb5_get_in_tkt}, \funcname{krb5_get_in_tkt_with_password}, and
\funcname{krb5_get_in_tkt_with_skey}; at some sites, the Kerberos server can be
configured so that during the initial ticket transation, it will only
return encrypted tickets after the user has proven his or her identity
using a supported preauthentication mechanism.  This is done to make
certain password guessing attacks more difficult to carry out.



\begin{center}
\begin{tabular}{lcc}
\multicolumn{1}{c}{Symbol}&In & Valid for \\
&RFC?&get_in_tkt? \\ \hline
KRB5_PADATA_NONE		& yes	& yes	\\
KRB5_PADATA_AP_REQ		& yes	&	\\
KRB5_PADATA_TGS_REQ		& yes	&	\\
KRB5_PADATA_PW_SALT		& yes	&	\\
KRB5_PADATA_ENC_TIMESTAMP	& yes	& yes	\\
KRB5_PADATA_ENC_SECURID		&	& yes	\\
\end{tabular}
\end{center}
\label{padata-types}

KRB5_PADATA_TGS_REQ is rarely used by a programmer; it is used to pass
the ticket granting ticket to the Ticket Granting Service (TGS) during a
TGS transaction (as opposed to an initial ticket transaction).

KRB5_PW_SALT is not really a preauthentication method at all.  It is
passed back from the Kerberos server to application program, and it
contains a hint to the proper password salting algorithm which should be
used during the initial ticket exchange.

%The encription type can also be specified in
%\funcname{krb5_get_in_tkt}, however normally only one keytype is used
%in any one database.
%
%\begin{center}
%\begin{tabular}{llc}
%\multicolumn{1}{c}{Symbol}&\multicolumn{1}{c}{RFC}& Supported? \\
%& \multicolumn{1}{c}{section} &  \\ \hline
%ETYPE_NULL		& 6.3.1	& 	\\
%ETYPE_DES_CBC_CRC	& 6.3.2	& yes	\\
%ETYPE_DES_CBC_MD4	& 6.3.3	&	\\
%ETYPE_DES_CBC_MD5	& 6.3.4	&	\\
%ETYPE_RAW_DES_CBC	&	& yes	\\
%\end{tabular}
%\end{center}
%\label{etypes}


