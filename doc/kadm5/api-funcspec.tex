\documentstyle[12pt,fullpage,changebar]{article}

% $Id$

\setlength{\parskip}{.7\baselineskip}
\setlength{\parindent}{0pt}

\def\secure{OV*Secure}
\def\v#1{\verb+#1+}

\title{OV*Secure Admin \\ Functional Specifications}
\author{}
\date{DRAFT --- \today}

%%%%%%%%%%%%%%%%%%%%%%%%%%%%%%%%%%%%%%%%%%%%%%%%%%%%%%%%%%%%%%%%%%%%%%
%% Make _ actually generate an _, and allow line-breaking after it.
\let\underscore=\_
\catcode`_=13
\def_{\underscore\penalty75\relax}
%%%%%%%%%%%%%%%%%%%%%%%%%%%%%%%%%%%%%%%%%%%%%%%%%%%%%%%%%%%%%%%%%%%%%%

\begin{document}

\maketitle

{\setlength{\parskip}{0pt}\tableofcontents}

\section{Policies and Password Quality}

The Admin API Password Quality mechanism provides the following
controls.  Note that two strings are defined to be ``significantly
different'' if they differ by at least two characters.

\begin{itemize}
\item A minimum length can be required; a password with
fewer than the specified number of characters will not be accepted.

\item A minimum number of character classes can be required; a
password that does not contain at least one character from at least
the specified number of character classes will not be accepted.  The
character classes are defined by islower(), isupper(), isdigit(),
ispunct(), and other.

\item Passwords can be required to be different from
previous passwords; a password that generates the same encryption key
as any of the principal's specified previous number of passwords will
not be accepted.  This comparision is performed on the encryption keys
generated from the passwords, not on the passwords themselves.

\item A single ``forbidden password'' dictionary can be specified for all
users; a password that is not significantly different from every word
in the dictionary will not be accepted.

\item A password that is not significantly different from each
component and the realm of the principal's name will not be accepted.
\end{itemize}

\section{Admin API}

This section describes the Admin API that can be used to maintain
principals and policies.  It describes the data structures used for
each function and the interpretation of each data type field, the
semantics of each API function, and the possible return codes.

The Admin API is intended to be used by remote clients using an RPC
interface.  It is implemented by the admin server running on the
Kerberos master server.  It may also be possible for a program running
on the Kerberos master server to use the Admin API directly, without
going through the admin server.

\subsection{Data Structures}

This section describes the data structures used by the Admin API that
are unique to \secure{}.  They are defined in ovsec_admin/admin.h.

\subsubsection{Principals, ovsec_kadm_principal_ent_t}
\label{sec:principal-structure}

A Kerberos principal entry is represented by a
ovsec_kadm_principal_ent_t.  It contains a subset of the information
stored in the master Kerberos database as well as the additional
information maintained by \secure{}.  In the current version, the only
additional information is the principal's policy and the
aux_attributes flags.

The principal may or may not have a policy enforced on it.  If the
POLICY bit (see section \ref{sec:masks}) is set in aux_attributes, the
policy field names the principal's policy.  If the POLICY bit is not
set in aux_attributes, no policy is enforced on the principal and the
value of the policy field is undefined.

\begin{figure}[htbp]
\begin{verbatim}
typedef struct _ovsec_kadm_principal_ent_t {
        krb5_principal principal;

        krb5_timestamp princ_expire_time;
        krb5_timestamp last_pwd_change;
        krb5_timestamp pw_expiration;
        krb5_deltat max_life;
        krb5_principal mod_name;
        krb5_timestamp mod_date;
        krb5_flags attributes;
        krb5_kvno kvno;
        krb5_kvno mkvno;

        char * policy;
        u_int32 aux_attributes;
} ovsec_kadm_principal_ent_rec, *ovsec_kadm_principal_ent_t;
\end{verbatim}
\caption{Definition of ovsec_kadm_principal_ent_t.}
\label{fig:princ-t}
\end{figure}

The fields of an ovsec_kadm_principal_ent_t are interpreted as
follows.

\begin{description}
\item[principal] The name of the principal; must conform to Kerberos
naming specifications.

\item[princ_expire_time] The expire time of the principal as a Kerberos
timestamp.  No Kerberos tickets will be issued for a principal after
its expire time.

\item[last_pwd_change] The time this principal's password was last
changed, as a Kerberos timestamp.

\item[pw_expiration] The expire time of the user's current password, as a
Kerberos timestamp.  No application service tickets will be issued for the
principal once the password expire time has passed.  Note that the
user can still obtain ticket-granting tickets.

\item[max_life] The maximum lifetime of any Kerberos ticket issued to
this principal.

\item[attributes] A bitfield of attributes for use by the KDC.  
Note that only some are explicitly supported by \secure{}.  XXX Note:
We really should check what all of these mean.

\begin{tabular}{clr}
{\bf Supported} & {\bf Name} & {\bf Value} \\
  & KRB5_KDB_DISALLOW_POSTDATED     & 0x00000001 \\
  & KRB5_KDB_DISALLOW_FORWARDABLE   & 0x00000002 \\
X & KRB5_KDB_DISALLOW_TGT_BASED     & 0x00000004 \\
  & KRB5_KDB_DISALLOW_RENEWABLE     & 0x00000008 \\
  & KRB5_KDB_DISALLOW_PROXIABLE     & 0x00000010 \\
  & KRB5_KDB_DISALLOW_DUP_SKEY      & 0x00000020 \\
X & KRB5_KDB_DISALLOW_ALL_TIX       & 0x00000040 \\
  & KRB5_KDB_REQUIRES_PRE_AUTH      & 0x00000080 \\
  & KRB5_KDB_REQUIRES_HW_AUTH       & 0x00000100 \\
X & KRB5_KDB_REQUIRES_PWCHANGE      & 0x00000200 \\
  & KRB5_KDB_DISALLOW_SVR           & 0x00001000 \\
  & KRB5_KDB_PWCHANGE_SERVICE       & 0x00002000
\end{tabular}

The interpretation of each bit is as follows.  For each of the bits
that disables a corresponding KDC_OPT option, the option is disabled
on an AS_REQ if the bit is set on either the client or the server, and
the option is disabled on TGS_REQ if the bit is set on the server (the
setting of the bit on the client is irrelevant for a TGS_REQ).

\begin{description}
\item[KRB5_KDB_DISALLOW_POSTDATED]  Disables KDC_OPT_ALLOW_POSTDATE
and KDC_OPT_POSTDATED on AS_REQ and TGS_REQ.

\item[KRB5_KDB_DISALLOW_FORWARDABLE] Disables KDC_OPT_FORWARDABLE on
for AS_REQ and TGS_REQ.

\item[KRB5_KDB_DISALLOW_TGT_BASED] All TGS_REQ requests will fail for
a principal with this bit set.

\item[KRB5_KDB_DISALLOW_RENEWABLE] Disables KDC_OPT_RENEWABLE for
AS_REQ and TGS_REQ.  

\item[KRB5_KDB_DISALLOW_PROXIABLE] Disables KDC_OPT_PROXIABLE on
AS_REQ and TGS_REQ.

\item[KRB5_KDB_DISALLOW_DUP_SKEY] Disables KDC_OPT_ENC_TKT_IN_SKEY on
TGS_REQ.

\item[KRB5_KDB_DISALLOW_ALL_TIX] All AS_REQ requests fail if this bit
is set for the client or the server, and all TGS_REQ requests fail if
this bit is set for the server.  Note that this bit can be set
automatically if the symbol KRBCONF_KDC_MODIFIES_KDC is defined and a
specified number of pre-authentication attempts fail.

\item[KRB5_KDB_REQUIRES_PRE_AUTH] Any AS_REQ will fail if this bit is
set and the padata field of the request is empty.  Any TGS_REQ will
fail if this bit is set and the TKT_FLAG_PRE_AUTH bit is not set in
the tgt.  Thus, it is possible to have the bit not set on the TGT but
to have a specific service require pre-authentication.

\item[KRB5_KDB_REQUIRES_HW_AUTH] Unclear.

\item[KRB5_KDB_REQUIRES_PWCHANGE] An AS_REQ will fail if this bit is
set on the client and the KRB5_KDC_PWCHANGE_SERVICE bit is not set on
the server.

\item[KRB5_KDB_DISALLOW_SVR] All AS_REQ and TGS_REQ request will fail
if the server has this bit set.

\item[KRB5_KDB_PWCHANGE_SERVICE] See KRB5_KDC_REQUIRES_PWCHANGE.
\end{description}

\item[mod_name] The name of the Kerberos principal that most recently
modified this principal.

\item[mod_date] The time this principal was last modified, as a Kerberos
timestamp.

\item[kvno] The version of the principal's current key.

\item[mkvno] The version of the Kerberos Master Key in effect when
this principal's key was last changed.

\item[policy] If the POLICY bit is set in aux_attributes, the name
of the policy controlling this principal.

\item[aux_attributes]  A bitfield of flags for use by the
administration system.  Currently, the only valid flag is POLICY, and
it indicates whether or not the principal has a policy enforced on it.
\end{description}

\subsubsection{Policies, ovsec_kadm_policy_ent_t}
\label{sec:policy-fields}

If the POLICY bit is set in aux_attributes, the \v{policy} name field
in the ovsec_kadm_principal_ent_t structure refers to a password
policy entry defined in a \v{ovsec_kadm_policy_ent_t}.

\begin{verbatim}
typedef struct _ovsec_kadm_policy_ent_t {
        char *policy;

        u_int32 pw_min_life;
        u_int32 pw_max_life;
        u_int32 pw_min_length;
        u_int32 pw_min_classes;
        u_int32 pw_history_num;
        u_int32 policy_refcnt;
} ovsec_kadm_policy_ent_rec, *ovsec_kadm_policy_ent_t;
\end{verbatim}

The fields of an ovsec_kadm_policy_ent_t are interpreted as follows.
Note that a policy's values only apply to a principal using that
policy.

\begin{description}
\item[policy] The name of this policy, as a NULL-terminated string.
The ASCII characters between 32 (space) and 126 (tilde), inclusive,
are legal.

\item[pw_min_life] The minimum password lifetime, in seconds.
A principal cannot change its password before pw_min_life seconds have
passed since last_pwd_change.

\item[pw_max_life] The default duration, in seconds, used to compute
pw_expiration when a principal's password is changed.

\item[pw_min_length] The minimum password length, in characters.  A
principal cannot set its password to anything with fewer than this
number of characters.

\item[pw_min_classes] The minimum number of character classes in the
password.  This value can only be 1, 2, 3, or 4.  A principal cannot
set its password to anything with fewer than this number of character
classes in it.

\item[pw_history_num] The number of past passwords that are
stored for the principal; its maximum value is 10.  A principal cannot
set its password to any of its previous pw_history_num passwords.

\item[policy_refcnt]  The number of principals currently using this policy.
A policy cannot be deleted unless this number is zero.
\end{description}

\subsubsection{Create/Modify Masks}
\label{sec:masks}

The API functions for creating and modifying principals and policies
allow for a relevant subset of the fields of the
ovsec_kadm_principal_ent_t and ovsec_kadm_policy_ent_t to be specified
or changed.  The chosen fields are determined by a bitmask that is
passed to the relevant function.  Each API function has different
rules for which mask values can be specified, and can specify whether
a given mask value is mandatory, optional, or forbidden.  Mandatory
fields must be present and forbidden fields must not be present or an
error is generated.  When creating a principal or policy, optional
fields have a default value if they are not specified; when modifying
a principal or policy, optional fields are unchanged if they are not
specified.

The masks for principals are in table \ref{tab:princ-bits} and the
masks for policies are in table \ref{tab:policy-bits}.  The
OVSEC_KADM_ prefix has been removed from the Name fields.  In the
Create and Modify fields, M means mandatory, F means forbidden, and O
means optional.  Create fields that are optional specify the default
value.

Note that the POLICY and POLICY_CLR bits are special.  When POLICY is
set, the policy is assigned to the principal.  When POLICY_CLR is
specified, the policy is unassigned to the principal and as a result
no policy controls the principal.

If the principal has a policy assigned, the POLICY bit is set in
aux_attributes.

\begin{table}[htbp]
\begin{tabular}{@{}lclll}
{\bf Name} & {\bf Value} & {\bf Field Affected} & {\bf Create} & 
        {\bf Modify} \\
PRINCIPAL               & 0x000001 & principal & M & F \\
PRINC_EXPIRE_TIME       & 0x000002 & princ_expire_time & O, K/M value & O \\
PW_EXPIRATION           & 0x000004 & pw_expiration & O, now+pw_max_life & O \\
LAST_PWD_CHANGE         & 0x000008 & last_pwd_change & F & F \\
ATTRIBUTES              & 0x000010 & attributes & O, 0 & O \\
MAX_LIFE                & 0x000020 & max_life & O, K/M value & O \\
MOD_TIME                & 0x000040 & mod_date & F & F \\
MOD_NAME                & 0x000080 & mod_name & F & F \\
KVNO                    & 0x000100 & kvno & O, 1 & O \\
MKVNO                   & 0x000200 & mkvno & F & F \\
AUX_ATTRIBUTES          & 0x000400 & aux_attributes & O, 0 & O \\
POLICY                  & 0x000800 & policy & O, none & O \\
POLICY_CLR              & 0x001000 & policy & F & O
\end{tabular}
\caption{Mask bits for creating/modifying principals.}
\label{tab:princ-bits}
\end{table}

\begin{table}[htbp]
\begin{tabular}{@{}lclll}
Name & Value & Field Affected & Create & Modify \\
POLICY                  & same     & policy & M & F \\
PW_MAX_LIFE             & 0x004000 & pw_max_life & O, 0 (infinite) & O \\
PW_MIN_LIFE             & 0x008000 & pw_min_life & O, 0 & O \\
PW_MIN_LENGTH           & 0x010000 & pw_min_length & O, 0 & O \\
PW_MIN_CLASSES          & 0x020000 & pw_min_classes & O, 1 & O \\
PW_HISTORY_NUM          & 0x040000 & pw_history_num & O, 0 & O \\
REF_COUNT               & 0x080000 & pw_refcnt & O, 0 & O 
\end{tabular}
\caption{Mask bits for creating/modifying policies.}
\label{tab:policy-bits}
\end{table}

\subsection{Constants}

Several values are fixed and known through \#define's in include files.  

These are defined in <ovsec_admin/admin.h>:

\begin{description}
\item[admin service principal] ADM_PRINCIPAL (``admin'')
\item[admin history key] HIST_PRINCIPAL (``admin/history'')
\item[server acl file path] ACLFILE (``/krb/ovsec_admin.acl'')
\end{description}

The location of the admin policy and principal databases are defined
in <ovsec_admin/adb.h>:

\begin{description}
\item[admin policy database] POLICY_DB (``/krb5/policy.db'')
\item[admin principal database] PRINCIPAL_DB (``/krb5/principal.db'')
\end{description}

\subsection{Error Codes}

The error codes that can be returned by admin functions are listed
below.  Error codes indicated with a ``*'' can be returned by every
admin function and always have the same meaning; these codes are
omitted from the list presented with each function.  

The admin system guarantees that a function that returns an error code
has no other side effect.

The Admin system will use \v{com_err} for error codes.  Note, this
means that \v{com_err} codes may returned from functions that the
admin routines call (e.g. the kerberos library). Callers should not
expect that only OVSEC errors will be returned.  The Admin system
error code table name will be ``kadm'', and the offsets will be the
same as the order presented here.

\begin{description}
\item[* OVSEC_KADM_OK] Operation successful.
\item[* OVSEC_KADM_FAILURE] Operation failed for unspecified reason.
\item[* OVSEC_KADM_AUTH_GET] Caller is not authorized to perform
operations requiring the ``get'' privilege.
\item[* OVSEC_KADM_AUTH_ADD] Caller is not authorized to perform
operations requiring the ``add'' privilege.
\item[* OVSEC_KADM_AUTH_MODIFY] Caller is not authorized to perform
operations requiring the ``modify'' privilege.
\item[* OVSEC_KADM_AUTH_DELETE] Caller is not authorized to perform
operations requiring the ``delete'' privilege.
\item[* OVSEC_KADM_BAD_ARG] Invalid arguments would result in a memory
error.
\item[* OVSEC_KADM_BAD_DB] A database inconsistency was detected.
\item[* OVSEC_KADM_MEM] Out of memory performing operation.
\item[OVSEC_KADM_DUP] The operation would create a duplicate principal or
policy.
\item[OVSEC_KADM_UNK_PRINC]  The named principal does not exist.
\item[OVSEC_KADM_UNK_POLICY] The named policy does not exist.
\item[OVSEC_KADM_BAD_MASK] The principal or policy field mask is invalid
for the current operation.
\item[OVSEC_KADM_BAD_CLASS] The number of character classes specified
is invalid.
\item[OVSEC_KADM_BAD_LENGTH] The specified minimum or maximum length
is invalid; minimum lengths must be non-negative and maximum lengths
must be positive.
\item[OVSEC_KADM_PASS_Q_TOOSHORT] The password does not contain enough
characters.
\item[OVSEC_KADM_PASS_Q_CLASS] The password must contain characters
from more character classes.
\item[OVSEC_KADM_PASS_Q_DICT] The password is in the password
dictionary.
\item[OVSEC_KADM_PASS_REUSE] The specified password is in the principal's
password history.
\item[OVSEC_KADM_PASS_TOOSOON] The current password's minimum lifetime
has not passed.
\item[OVSEC_KADM_POLICY_REF] The named policy's refcnt is not zero.
\end{description}

\subsection{Authorization}

Each Admin API function requires a specific authorization to run.
This version uses a simple named privilege system with the following
names and meanings:

\begin{description}
\item[Get] Able to examine the attributes (NOT key data) of principals
and policies. 
\item[Add] Able to add principals and policies.
\item[Modify] Able to modify attributes of existing principals and policies.
\item[Delete] Able to remove principals and policies.
\end{description}

Privileges are specified via an external configuration file on the
Kerberos master server (see section \ref{sec:acls}).

Each API function description identifies the privilege required to
perform it.

\subsection{Function Overview}

The functions provided by the Admin API, and the authorization they
require, are listed in the table below.  The ``ovsec_kadm_'' prefix
has been removed from each function name.

The function semantics in the following sections omit details that are
the same for every function.

\begin{itemize}
\item The effects of every function are atomic.

\item Every function performs an authorization check and returns
the appropriate OVSEC_KADM_AUTH_* error code if the caller does not
have the required privilege.  No other information or error code is
ever returned to an unauthorized user.

\item Every function checks its arguments for NULL pointers or other
obviously invalid values, and returns OVSEC_KADM_BAD_ARG if any are
detected.

\item Any function that performs a policy check uses the policy named
in the principal's policy field.  If the POLICY bit is not set in the
principal's aux_attributes field, however, the principal has no
policy, so the policy check is not performed.

\item Unless otherwise specified, all functions return OVSEC_KADM_OK.
\end{itemize}

\begin{tabular}{@{}llp{3.24in}}
{\bf Function Name} & {\bf Authorization} & {\bf Operation} \\

create_principal & add & Create a new principal. \\
delete_principal & delete & Delete a principal. \\
modify_principal & modify & Modify the attributes of an existing
        principal (not password). \\
rename_principal & add and delete & Rename a principal. \\
get_principal & get\footnotemark & Retrieve a principal. \\
chpass_principal & modify\footnotemark[\thefootnote] &
         Change a principal's password. \\
randkey_principal & modify\footnotemark[\thefootnote] &
        Randomize a principal's key. \\
create_policy & add & Create a new policy. \\
delete_policy & delete & Delete a policy. \\
modify_policy & modify & Modify the attributes of a policy. \\
get_policy & get & Retrieve a policy. \\
free_principal_ent & none & Free the memory associated with an
                ovsec_kadm_principal_ent_t. \\
free_policy_ent & none & Free the memory assocated with an
                ovsec_kadm_policy_ent_t. \\
get_privs & none & Return the caller's admin server privileges.
\end{tabular}
\footnotetext[\thefootnote]{These functions also allow a principal to
perform the operation on itself; see the function's semantics for
details.}

\subsection{ovsec_kadm_create_principal}

\begin{verbatim}
ovsec_kadm_ret_t
ovsec_kadm_create_principal(ovsec_kadm_principal_ent_t princ, u_int32 mask,
		char *pw, int override_qual);
\end{verbatim}

AUTHORIZATION REQUIRED: add

\begin{enumerate}
\item Determine whether password quality checks should be overriden.
\begin{enumerate}
\item If the POLICY bit is not set in aux_attributes, set
override_qual to true.
\item Otherwise, use the specified override_qual.
\end{enumerate}

\item Return OVSEC_KADM_BAD_MASK if the mask is invalid.
\item If the named principal exists, return OVSEC_KADM_DUP.
\item If the POLICY bit is set and the named policy does not exist,
return OVSEC_KADM_UNK_POLICY.
\item If override_qual is false and the password does not meet the
quality standards, return the appropriate OVSEC_KADM_PASS_Q_* error
code.
\item Store the principal, set the key.  The key is generated with
Kerberos' string-to-key function, using the salt method specified on
the admin server's command line; see section \ref{sec:commandline}.
\item If the POLICY bit is set, increment the named policy's reference
count by one.

\item Set the pw_expiration field.
\begin{enumerate}
\item If the POLICY bit is not set, then
\begin{enumerate}
\item if the PW_EXPIRATION bit is set, set pw_expiration to the given
value, else
\item set pw_expiration to never.
\end{enumerate}
\item Otherwise, if the PW_EXPIRATION bit is set, set pw_expiration to
the sooner of the given value and now + pw_max_life.
\item Otherwise, set pw_expiration to now + pw_max_life.
\end{enumerate}

\item Set mod_date to now and set mod_name to caller.
\item Set last_pwd_change to never.
\end{enumerate}

RETURN CODES:

\begin{description}
\item[OVSEC_KADM_BAD_MASK] The field mask is invalid for a create
operation.
\item[OVSEC_KADM_DUP] Principal already exists.
\item[OVSEC_KADM_UNK_POLICY] Policy named in entry does not exist.
\item[OVSEC_KADM_PASS_Q_*] Specified password does not meet policy
standards.
\end{description}

\subsection{ovsec_kadm_delete_principal}

\begin{verbatim}
ovsec_kadm_ret_t
ovsec_kadm_delete_principal(krb5_principal);
\end{verbatim}

AUTHORIZATION REQUIRED: delete 

\begin{enumerate}
\item Return OVSEC_KADM_UNK_PRINC if the principal does not exist.
\item If the POLICY bit is set in aux_attributes, decrement the named
policy's reference count by one.
\item Delete principal.
\end{enumerate}

RETURN CODES: 

\begin{description}
\item[OVSEC_KADM_UNK_PRINC] Principal does not exist.
\end{description}

\subsection{ovsec_kadm_modify_principal}

\begin{verbatim}
ovsec_kadm_ret_t
ovsec_kadm_modify_principal(ovsec_kadm_principal_ent_t, u_int32);
\end{verbatim}

Modify the attributes of the principal named in
ovsec_kadm_principal_ent_t. This does not allow the principal to be
renamed or for its password to be changed.

AUTHORIZATION REQUIRED: modify

\begin{enumerate}
\item Return OVSEC_KADM_UNK_PRINC if the principal does not exist.
\item Return OVSEC_KADM_BAD_MASK if the mask is invalid.
\item If POLICY bit is set but the new policy does not exist, return
OVSEC_KADM_UNK_POLICY.
\item If either the POLICY or POLICY_CLR bits are set, update the
corresponding bits in aux_attributes.

\item Update policy reference counts.
\begin{enumerate}
\item If the POLICY bit is set, then increment policy count on new
policy.
\item If the POLICY or POLICY_CLR bit is set, and the POLICY bit in
aux_attributes is set, decrement policy count on old policy.
\end{enumerate}

\item Set pw_expiration according to the new policy.
\begin{enumerate}
\item If the POLICY bit is not set in aux_attributes, then
\begin{enumerate}
\item if the PW_EXPIRATION bit is set, set pw_expiration to the given
value, else
\item set pw_expiration to never.
\end{enumerate}
\item Otherwise, if the PW_EXPIRATION bit is set, set pw_expiration to
the sooner of the given value and last_pwd_change + pw_max_life.
\item Otherwise, set pw_expiration to last_pwd_change + pw_max_life.
\end{enumerate}

\item Update the fields specified in the mask.
\item Update mod_name field to caller and mod_date to now.
\end{enumerate}

RETURN CODES:

\begin{description}
\item[OVSEC_KADM_UNK_PRINC] Entry does not exist.
\item[OVSEC_KADM_BAD_MASK] The mask is not valid for a modify
operation.
\item[OVSEC_KADM_UNK_POLICY] The POLICY bit is set but the new
policy does not exist.
\end{description}

\subsection{ovsec_kadm_rename_principal}

\begin{verbatim}
ovsec_kadm_ret_t
ovsec_kadm_rename_principal(krb5_principal source, krb5_principal target);
\end{verbatim}

AUTHORIZATION REQUIRED: add and delete

\begin{enumerate}
\item Check to see if source principal exists, if not return
OVSEC_KADM_UNK_PRINC error. 
\item Check to see if target exists, if so return OVSEC_KADM_DUP error.
\item Rename principal.
\end{enumerate}

Note that since the principal name may have been used as the salt for
the principal's key, renaming the principal may render the principal's
current password useless; with the new salt, the key generated by
string-to-key on the password will suddenly be different.  Therefore,
an application that renames a principal must also require the user to
specify a new password for the principal (and administrators should
notify the affected party).

Note also that, by the same argument, renaming a principal will
invalidate that principal's password history information; since the
salt will be different, a user will be able to select a previous
password without error.

RETURN CODES: 

\begin{description}
\item[OVSEC_KADM_UNK_PRINC] Source principal does not exist.
\item[OVSEC_KADM_DUP] Target principal already exist.
\end{description}

\subsection{ovsec_kadm_chpass_principal}

\begin{verbatim}
ovsec_kadm_ret_t
ovsec_kadm_chpass_principal(krb5_principal princ, char *pw, 
                int override_qual);
\end{verbatim}

AUTHORIZATION REQUIRED: modify, or the calling principal being the
same as the princ argument.

Change a principal's password.  

In the description below, all the checks that can result in
policy-related errors do not apply to callers that have the modify
privilege but are {\it not} the same as the principal being affected.
Thus, an administrator can change a principal's password in violation
of that principal's policy, but cannot change its own password in
violation of its own policy.

Note that the policy checks are only be performed if the POLICY bit is
set in the principal's aux_attributes field.

\begin{enumerate}
\item Determine whether password quality checks should be overridden.
\begin{enumerate}
\item If the POLICY bit is not set in aux_attributes, set
override_qual to true.
\item Otherwise, if the caller does not have the modify priviledge,
set override_qual to false.
\item Otherwise, if the caller has the modify privilege, but princ is the
same as the caller, set override_qual to false.
\item Otherwise, if the caller has the modify privilege and princ is
not the same as the caller, leave override_qual as it is.
\end{enumerate}
\item Make sure principal exists, if not return OVSEC_KADM_UNK_PRINC error.
\item If override_qual is false and (now - last_pwd_change) $<$
pw_min_life, return OVSEC_KADM_PASS_TOOSOON.
\item If override_qual is false and the password does not meet the quality
standards, return the appropriate OVSEC_KADM_PASS_Q_* error code.
\item Convert password to key.  The key is generated with
Kerberos' string-to-key function, using the salt method specified on
the admin server's command line; see section \ref{sec:commandline}.
\item If override_qual is false and the new key is in the principal's
password history, return OVSEC_KADM_PASS_REUSE.
\item Store old key in history.
\item Update principal to have new key.
\item Increment principal's key version number by one.
\item If the POLICY bit is set, set pw_expiration to now + max_pw_life.
\item Update last_pwd_change and mod_date to now, update mod_name to
caller.
\end{enumerate}

RETURN CODES:

\begin{description}
\item[OVSEC_KADM_UNK_PRINC] Principal does not exist.
\item[OVSEC_KADM_PASS_Q_*] Requested password does not meet quality
standards. 
\item[OVSEC_KADM_PASS_REUSE] Requested password is in user's
password history. 
\item[OVSEC_KADM_PASS_TOOSOON] Current password has not reached minimum
life. 
\end{description}

\subsection{ovsec_kadm_randkey_principal}

\begin{verbatim}
ovsec_kadm_ret_t
ovsec_kadm_randkey_principal(krb5_principal, krb5_keyblock **,
		int override_qual);
\end{verbatim}

Generate and assign a new random key to the named principal, and
return the generated key in allocated storage.  The caller must free
the returned krb5_keyblock * with krb5_free_keyblock.

AUTHORIZATION REQUIRED: modify, or the calling principal being the
same as the princ argument.

In the description below, all the checks that can result in
key-related errors do not apply to callers that have the modify
privilege but are {\it not} the same as the principal being affected.
Thus, an administrator can randomize a principal's password in
violation of the principal's policy, but cannot randomize its own
password in violation of its own policy.

Note that the policy checks are only be performed if the POLICY bit is
set in the principal's aux_attributes field.

\begin{enumerate}
\item Determine whether policy checks should be overriden.
\begin{enumerate}
\item If the POLICY bit is not set in aux_attributes, set
override_qual to true.
\item Otherwise, if the caller does not have the modify priviledge,
set override_qual to false.
\item Otherwise, if the caller has the modify privilege, but princ is the
same as the caller, set override_qual to false.
\item Otherwise, if the caller has the modify privilege and princ is
not the same as the caller, leave override_qual as it is.
\end{enumerate}
\item If the principal does not exist, return OVSEC_KADM_UNK_PRINC.
\item If override_qual is false and (now - last_pwd_change) $<$
pw_min_life, return OVSEC_KADM_PASS_TOOSOON.
\item Store old key in history.
\item Update principal to have new key.
\item Increment principal's key version number by one.
\item If the POLICY bit in aux_attributes is set, set pw_expiration to
now + max_pw_life.
\item Update last_pwd_change and mod_date to now, update mod_name to
caller.
\end{enumerate}

RETURN CODES:

\begin{description}
\item[OVSEC_KADM_UNK_PRINC] Principal does not exist.
\item[OVSEC_KADM_PASS_TOOSOON] The minimum lifetime for the current
key has not expired.
\end{description}

This function can also be used as part of a sequence to create a new
principal with a random key.  The steps to perform the operation
securely are

\begin{enumerate}
\item Create the principal with ovsec_kadm_create_principal with a
random password string and with the KRB5_KDB_DISALLOW_ALL_TIX bit set
in the attributes field.

\item Randomize the principal's key with ovsec_kadm_randkey_principal.

\item Call ovsec_kadm_modify_principal to reset the
KRB5_KDB_DISALLOW_ALL_TIX bit in the attributes field.
\end{enumerate}

\subsection{ovsec_kadm_get_principal}

\begin{verbatim}
ovsec_kadm_ret_t
ovsec_kadm_get_principal(krb5_principal princ, ovsec_kadm_principal_ent_t *ent);  
\end{verbatim}

Return the principal's attributes in allocated memory.  The caller
must free the returned entry with ovsec_kadm_free_principal_ent.

AUTHORIZATION REQUIRED: get, or the calling principal being the same
as the princ argument.

RETURN CODES:

\begin{description}
\item[OVSEC_KADM_UNK_PRINC] Principal does not exist.
\end{description}

\subsection{ovsec_kadm_create_policy}

\begin{verbatim}
ovsec_kadm_ret_t
ovsec_kadm_create_policy(ovsec_kadm_policy_ent_t, u_int32); 
\end{verbatim}

Create a new policy.

AUTHORIZATION REQUIRED: add

\begin{enumerate}
\item Check to see if mask is valid, if not return OVSEC_KADM_BAD_MASK error.
\item Check to see if the policy already exists, if so return
OVSEC_KADM_DUP error. 
\item If the PW_MIN_CLASSES bit is set and pw_min_classes is not 1, 2,
3, or 4, return OVSEC_KADM_BAD_CLASS.
\item Create a new policy setting the appropriate fields determined
by the mask.
\end{enumerate}

RETURN CODES:

\begin{description}
\item[OVSEC_KADM_DUP] Policy already exists
\item[OVSEC_KADM_BAD_MASK] The mask is not valid for a create
operation.
\item[OVSEC_KADM_BAD_CLASS] The specified number of character classes
is invalid.
\end{description}

\subsection{ovsec_kadm_delete_policy}

\begin{verbatim}
ovsec_kadm_ret_t
ovsec_kadm_delete_policy(char *);
\end{verbatim}

Deletes a policy.

AUTHORIZATION REQUIRED: delete

\begin{enumerate}
\item Return OVSEC_KADM_UNK_POLICY if the named policy does not exist.
\item Return OVSEC_KADM_POLICY_REF if the named policy's refcnt is not 0.
\item Delete policy.
\end{enumerate}

RETURN CODES:

\begin{description}
\item[OVSEC_KADM_UNK_POLICY] Policy does not exist.
\item[OVSEC_KADM_POLICY_REF] Policy is being referenced. 
\end{description}

\subsection{ovsec_kadm_modify_policy}

\begin{verbatim}
ovsec_kadm_ret_t
ovsec_kadm_modify_policy(ovsec_kadm_policy_ent_t, u_int32);
\end{verbatim}

Modify an existing policy.  Note that modifying a policy has no affect
on a principal using the policy until the next time the principal's
password is changed.

AUTHORIZATION REQUIRED: modify

\begin{enumerate}
\item Check to see if mask is legal, if not return OVSEC_KADM_BAD_MASK error.
\item Check to see if policy exists, if not return
OVSEC_KADM_UNK_POLICY error.
\item If the PW_MIN_CLASSES bit is set and pw_min_classes is not 1, 2,
3, or 4, return OVSEC_KADM_BAD_CLASS.
\item Update the fields specified in the mask.
\end{enumerate}

RETURN CODES: 

\begin{description}
\item[OVSEC_KADM_UNK_POLICY] Policy not found.
\item[OVSEC_KADM_BAD_MASK] The mask is not valid for a modify
operation.
\item[OVSEC_KADM_BAD_CLASS] The specified number of character classes
is invalid.
\end{description}

\subsection{ovsec_kadm_get_policy}

\begin{verbatim}
ovsec_kadm_ret_t
ovsec_kadm_get_policy(char *, ovsec_kadm_policy_ent_t *); 
\end{verbatim}

AUTHORIZATION REQUIRED: get

Return the policy's attributes in allocated memory.  The caller must
free the returned entry with ovsec_kadm_free_policy_ent.

RETURN CODES: 

\begin{description}
\item[OVSEC_KADM_UNK_POLICY] Policy not found.
\end{description}

\subsection{ovsec_kadm_free_principal_ent, _policy_ent}

\begin{verbatim}
ovsec_kadm_ret_t
ovsec_kadm_free_principal_ent(ovsec_kadm_principal_ent_t *);
\end{verbatim}

Free the memory that was allocated by a call to
ovsec_kadm_get_principal. 

AUTHORIZATION REQUIRED: none (local operation)

\begin{verbatim}
ovsec_kadm_ret_t
ovsec_kadm_free_policy_ent(ovsec_kadm_policy_ent_t *);
\end{verbatim}

Free memory that was allocated by a call to ovsec_kadm_get_policy.

AUTHORIZATION REQUIRED: none (local operation)

\subsection{ovsec_kadm_get_privs}

\begin{verbatim}
ovsec_kadm_ret_t
ovsec_kadm_get_privs(u_int32 *);
\end{verbatim}

Return the caller's admin server privileges in the integer pointed to
by the argument.  The Admin API does not define any way for a
principal's privileges to be set.  Note that this function will
probably be removed or drastically changed in future versions of this
system.

The returned value is a bitmask indicating the caller's privileges:

\begin{tabular}{llr}
{\bf Privilege} & {\bf Symbol} & {\bf Value} \\
Get & OVSEC_KADM_PRIV_GET & 0x01 \\
Add & OVSEC_KADM_PRIV_ADD & 0x02 \\
Modify & OVSEC_KADM_PRIV_MODIFY & 0x04 \\
Delete & OVSEC_KADM_PRIV_DELETE & 0x08
\end{tabular}

There is no guarantee that a caller will have a privilege indicated by
this function for any length of time; applications using this function
must still be prepared to handle all possible OVSEC_KADM_AUTH_* error
codes.

\section{Server}

The Admin API will be implemented by a server process running on the
same machine as the Kerberos server, and a client library to
communicate with the server.

\subsection{Command Line}
\label{sec:commandline}

The command line syntax of the admin server is

\begin{verbatim}
admin_server [-createsalt normal|none] [-modifysalt normal|none|keep]
\end{verbatim}

The -createsalt and -modifysalt arguments control the type of salt
used when creating and modifying keys in the Kerberos database,
respectively.  ``normal'' means the standard V5 salt which uses the
principal and realm name.  ``none'' means no salt, which is compatible
with Kerberos V4.  ``keep'' means maintain the previous salt when a
key is changed.

\subsection{Authorization ACLs}
\label{sec:acls}

The admin server will use a simple ACL mechanism to grant privileges
to principals.  The file {\tt /krb5/ovsec_admin_acl} will contain a
list of principals and their privileges.  It is read at start-up, and
can only be reread by restarting the admin server.

The format of this file is:

\begin{itemize}
\item Blank lines or lines beginning with ``\#'' are ignored.

\item ACL entry lines contain two fields separated by any number of
spaces or tabs.  The first field is a Kerberos name that can
optionally have ``*'' as any component, and the second field is the
privilege list.

\item The privilege list can contain a comma separated list of the
words ``get'', ``add'', ``modify'', and ``delete''.
\end{itemize}

The principal named in the first field of each ACL entry has the
privileges listed in the second field the ACL entry.  A principal
component name of ``*'' is a wildcard and matches any single
component; therefore, the name ``*/admin@FNORD.COM'' matches any
two-component principal name whose second component is ``admin'' in
the FNORD.COM realm.

\subsection{Logging}

The server will log each Admin API request.  The information logged
will include the authentication name of the caller, the operation
performed, the arguments to the operation, and the return value.
Unauthorized requests will be identifiable by the OVSEC_KADM_AUTH
error code.

\section{Tools}

Three tools will be provided to create and manage the admin databases.
This need only run on the admin server machine and do not need to
operate remotely.  The tools are:

\begin{description}
\item[ovsec_adm_create] create the admin service principal, the admin
history principal, and empty admin policy database, and an admin
principal database with an empty entry for every exist principal.
\item[ovsec_adm_db_export/import] dump or load the admin policy and
principal databases
\item[ovsec_adm_check] check the kdc and admin databases for
inconsistencies and repair them.
\end{description}

The details of these tools are described in their own documents.

\end{document}
