The key table functions deal with storing and retrieving service keys
for use by unattended services which participate in authentication exchanges.

Keytab routines should all be atomic.  Every routine that acquires
a non-sharable resource must release it before it returns.  For
example, an implementation is not allowed to leave a file open
for writing or to have a lock on a file.

Any keytab implementations must support multiple concurrent sequential scans.

The order of values returned from \funcname{krb5_kt_next_entry} is
unspecified and may be sorted to the implementor's convenience.

Although the ``right thing'' should happen if the program aborts
abnormally, a close routine is provided for freeing resources,
etc.  People should use the close routine when they are
finished.

\subsubsection{Per-type functions}
The following entry points must be implemented for each type of
key table; however, application programs are not expected to call
\funcname{krb5_kt_resolve_internal}, \funcname{krb5_kt_remove_internal},
or \funcname{krb5_kt_add_internal} directly.

In order to reduce the size of binary programs when static linking is
used, it is common to provide two \datatype{krb5_kt_ops} structures for
each key table type, one for reading only in which the pointers to the
add and delete functions are zero, and one for reading and writing.

\begin{funcdecl}{krb5_kt_resolve_internal}{krb5_error_code}{\funcin}
\funcarg{char *}{residual}
\funcout
\funcarg{krb5_keytab *}{id}
\end{funcdecl}

Fills in \funcparam{*id} with a handle identifying the keytab with name
``residual''.  The interpretation of ``residual'' is dependent on the
type of keytab.

\begin{funcdecl}{krb5_kt_get_name}{krb5_error_code}{\funcin}
\funcarg{krb5_keytab}{id}
\funcout
\funcarg{char *}{name,}
\funcin
\funcarg{int}{namesize}
\end{funcdecl}

\funcarg{name} is filled in with the first \funcparam{namesize} bytes of
the name of the keytab identified by \funcname{id}.
If the name is shorter than \funcparam{namesize}, then \funcarg{name}
will be null-terminated.

\begin{funcdecl}{krb5_kt_close}{krb5_error_code}{\funcin}
\funcarg{krb5_keytab}{id}
\end{funcdecl}

Closes the keytab identified by \funcparam{id} and invalidates
\funcparam{id}, and releases any other resources acquired during use of
the key table.

Requires that \funcparam{id} identifies a valid credentials cache.

\begin{funcdecl}{krb5_kt_get_entry}{krb5_error_code}{\funcin}
\funcarg{krb5_keytab}{id}
\funcarg{krb5_principal}{principal}
\funcarg{krb5_kvno}{vno}
\funcout
\funcarg{krb5_keytab_entry *}{entry}
\end{funcdecl}

Searches the keytab identified by \funcparam{id} for an entry whose
principal matches \funcparam{principal} and
whose key version number matches \funcparam{vno}.  If \funcparam{vno} is
zero, the first entry whose principal matches is returned.

Returns an error code if no suitable entry is found.  If an entry is
found, the entry is returned in \funcparam{*entry}; its contents should
be deallocated by calling \funcname{krb5_kt_free_entry} when no longer
needed.

\begin{funcdecl}{krb5_kt_free_entry}{krb5_error_code}{\funcinout}
\funcarg{krb5_keytab_entry *}{entry}
\end{funcdecl}

Releases all storage allocated for \funcparam{entry}, which must point
to a structure previously filled in by \funcname{krb5_kt_get_entry} or
\funcname{krb5_kt_next_entry}.

\begin{funcdecl}{krb5_kt_start_seq_get}{krb5_error_code}{\funcin}
\funcarg{krb5_keytab}{id}
\funcout
\funcarg{krb5_kt_cursor *}{cursor}
\end{funcdecl}

Prepares to read sequentially every key in the keytab identified by
\funcparam{id}.
\funcparam{cursor} is filled in with a cursor to be used in calls to
\funcname{krb5_kt_next_entry}.

\begin{funcdecl}{krb5_kt_next_entry}{krb5_error_code}{\funcin}
\funcarg{krb5_keytab}{id}
\funcout
\funcarg{krb5_keytab_entry *}{entry}
\funcinout
\funcarg{krb5_kt_cursor}{cursor}
\end{funcdecl}

Fetches the ``next'' entry in the keytab, returning it in
\funcparam{*entry}, and updates \funcparam{*cursor} for the next
request.  If the keytab changes during the sequential get, an error is
guaranteed.  \funcparam{*entry} should be freed after use by calling
\funcname{krb5_kt_free_entry}.

Requires that \funcparam{id} identifies a valid credentials cache.  and
\funcparam{*cursor} be a cursor returned by
\funcname{krb5_kt_start_seq_get} or a subsequent call to
\funcname{krb5_kt_next_entry}.

Errors: error code if no more cache entries or if the keytab changes.

\begin{funcdecl}{krb5_kt_end_seq_get}{krb5_error_code}{\funcin}
\funcarg{krb5_keytab}{id}
\funcarg{krb5_kt_cursor}{cursor}
\end{funcdecl}

Finishes sequential processing mode and invalidates \funcparam{cursor},
which must never be re-used after this call.

Requires that \funcparam{id} identifies a valid credentials cache.  and
\funcparam{*cursor} be a cursor returned by
\funcname{krb5_kt_start_seq_get} or a subsequent call to
\funcname{krb5_kt_next_entry}.

May return error code if \funcparam{cursor} is invalid.

\begin{funcdecl}{krb5_kt_remove_internal}{krb5_error_code}{\funcin}
\funcarg{krb5_keytab}{id}
\funcarg{krb5_keytab_entry *}{entry}
\end{funcdecl}

Searches the keytab \funcparam{id} for an entry that exactly matches
\funcparam{entry}.  If one is found, it is removed from the keytab.

Returns an error code if not found.

	
\begin{funcdecl}{krb5_kt_add_internal}{krb5_error_code}{\funcin}
\funcarg{krb5_keytab}{id}
\funcarg{krb5_keytab_entry *}{entry}
\end{funcdecl}

Stores \funcparam{entry} in the keytab \funcparam{id}.
Fails if the entry already exists.

This operation must, within the constraints of the operating system, not
return until it can verify that the write has completed succesfully.
For example, in a UNIX file-based implementation, this routine (or the part
of the underlying implementation that it calls) would be responsible for
doing an \funcname{fsync} on the file before returning.

Returns an error code if \funcparam{entry} is already present in the
keytab or if the key could not be stored (quota problem, etc).


\subsubsection{Glue functions}
The following functions are implemented in the base library and serve to
glue together the various types of key tables.

\begin{funcdecl}{krb5_kt_register}{krb5_error_code}{\funcin}
\funcarg{krb5_kt_ops *}{ops}
\end{funcdecl}


Adds a new ticket cache type to the set recognized by
\funcname{krb5_kt_resolve}.
Requires that a keytab type named \funcparam{ops{\ptsto}prefix} is not
yet known.

An error is returned if \funcparam{ops{\ptsto}prefix} is already known.

\begin{funcdecl}{krb5_kt_resolve}{krb5_error_code}{\funcin}
\funcarg{char *}{string_name}
\funcout
\funcarg{krb5_keytab *}{id}
\end{funcdecl}

Fills in \funcparam{*id} with a handle identifying the keytab with name
``string_name''.  The keytab is not opened.
Requires that \funcparam{string_name} be of the form ``type:residual'' and
``type'' is a type known to the library.

Errors: badly formatted name.
		
\begin{funcdecl}{krb5_kt_default_name}{krb5_error_code}{\funcin}
\funcarg{char *}{name}
\funcarg{int}{namesize}
\end{funcdecl}

\funcparam{name} is filled in with the first \funcparam{namesize} bytes of
the name of the default keytab.
If the name is shorter than \funcparam{namesize}, then the remainder of
\funcparam{name} will be zeroed.


\begin{funcdecl}{krb5_kt_default}{krb5_error_code}{\funcin}
\funcarg{krb5_keytab *}{id}
\end{funcdecl}

Fills in \funcparam{id} with  a handle identifying the default keytab.

\begin{funcdecl}{krb5_kt_read_service_key}{krb5_error_code}{\funcin}
\funcarg{krb5_pointer}{keyprocarg}
\funcarg{krb5_principal}{principal}
\funcarg{krb5_kvno}{vno}
\funcout
\funcarg{krb5_keyblock **}{key}
\end{funcdecl}

This function is suitable for use as a parameter to
\funcname{krb5_rd_req}.

If \funcname{keyprocarg} is not NULL, it is taken to be a
\datatype{char *} denoting the name of a keytab.  Otherwise, the default
keytab will be used.
The keytab is opened and searched for the entry identified by
\funcparam{principal} and \funcparam{vno}, returning the resulting key
in \funcparam{*key} or returning an error code if it is not found. 

\funcname{krb5_free_keyblock} should be called on \funcparam{*key} when
the caller is finished with the key.

Returns an error code if the entry is not found.

\begin{funcdecl}{krb5_kt_add_entry}{krb5_error_code}{\funcin}
\funcarg{krb5_keytab}{id}
\funcarg{krb5_keytab_entry *}{entry}
\end{funcdecl}

Calls the keytab-specific add routine \funcname{krb5_kt_add_internal}
with the same function arguments.  If this routine is not available,
then KRB5_KT_NOWRITE is returned.

\begin{funcdecl}{krb5_kt_remove_entry}{krb5_error_code}{\funcin}
\funcarg{krb5_keytab}{id}
\funcarg{krb5_keytab_entry *}{entry}
\end{funcdecl}

Calls the keytab-specific remove routine
\funcname{krb5_kt_remove_internal} with the same function arguments.
If this routine is not available, then KRB5_KT_NOWRITE is returned.
